\documentclass{report}
\usepackage[utf8]{inputenc} 
\usepackage{amsmath}
\usepackage{amsthm}
\usepackage{amssymb}


\begin{document}

\chapter{Chapter 29}

\section{Alternating-Current Circuits}
\subsection{Conceptual Problems}

\paragraph{}
1 - An ac generator rotates at 60 Hz and has a coil. Determine the time that elapses between successive peak emf values of the coil?\\
Successive peaks are one-half period apart. Hence the elapsed time between the peaks is $\frac{1}{2}T = \frac{1}{2f} = \frac{1}{2(60^{-1})} = \fbox{8.33 ms} $.

\paragraph{}
2 - When the rms voltage in an ac circuit doubles, the peak voltage is:\\
(a) doubled, (b) halved, (c) increased by a factor of $\sqrt{2}$, (d) not changed.\\
A relationship between $V$ and $V_{peak}$ can be used to determine the effect of doubling the rms voltage on the peak voltage.\\
First express the initial rms voltage in terms of the peak voltage:
$$V_{rms} = \frac{V_{peak}}{\sqrt{2}}$$
Then expres the doubled rms voltage in terms of the new peak voltage $V_{peak}^\prime$:
$$2V_{rms} = \frac{V_{peak}^\prime}{\sqrt{2}}$$
After you have done that you have to divide the second of these equations by the first and simplify to obtain:
$$\frac{2V_{rms}}{V_{rms}} = \frac{\frac{V_{peak}^\prime}{\sqrt{2}}}{\frac{V_{peak}}{\sqrt{2}}} \Rightarrow 2 = \frac{V_{peak}^\prime}{V_{peak}}$$
Now you only have to solve for $V_{peak}^\prime$:
$$V_{peak}^\prime = 2 V_{peak} \Rightarrow \fbox{(a)} \text{ is the correct answer.}$$
\paragraph{}
3 - Suppose the frequency in the circuit shown in figure 29-27 is doubled, then the inductance in the inductor will (a) double, (b) not change, (c) halve, (d) quadruple.\\
The inductance of an inductor is independent of the frequency of the circuit and is determined only by its construction. The reactance is dependent on the frequency, hence \fbox{(b)} is correct.

\paragraph{}
4 - Suppose the frequency in the circuit shown in figure 29-27 is doubled, then the inductive reactance of the inductor will (a) double, (b) not change, (c) halve, (d) quadruple.\\
The inductive reactance of an inductor and the frequency are related by $X_L = \omega L$. Therefore, when $\omega$ is doubled, $X_L$ will double as well. \fbox{(a)} is correct.

\paragraph{}
5 - Suppose the frequency in the circuit shown in figure 29-28 is doubled, then the capacitiive reactance of the circuit will (a) double, (b) not change, (c) halve, (d) quadruple.\\
The capacitive reactance of a capacitor varies with the frequency and are related by $X_C = \frac{1}{\omega C}$. Therefore, if $\omega$ change to its double then $X_C$ will halve. \fbox{(c)} is correct.

\paragraph{}
6 - (a) There is a circuit consisting of an ac generator and and ideal inductor, do exist time intervals when the inductor receives energy from the generator? When this happens? (b) Are there time intervals when the inductor supplies energy back to the generator? When this happens? Explain your answers.\\
The answer to both questions is yes. (a) During current magnitude increase in the inductor, the inductor absorbs power from the generator. (b) During current decrease in the inductor, the inductor supplies power to the generator.

\paragraph{}
7 - (a) Imagine a circuit consisting of a generator and a capacitor, are there any time intervals when the capacitor receives energy from the generator? When this happens? (b) Does the capacitor supplies power to the generator? When this happens? Explain your answers.\\
(a) The capacitor absorbs power from the generator while the magnitude of the charge is accumulating on either plate of the capacitor. (b) The capacitor supplies power to the generator whenever the magnitude of the charge on either plate of the capacitor is decreasing. 

\paragraph{}
8 - (a) Demonstrate that the SI unit of inductance multiplied by the SI unit of capacitance is equivalent to seconds squared. (b) Show that the SI unit of inductance divided by the SI unit of resistance is equivalent to seconds.\\
(a) First substitute the SI units of inductance and capacitance then simplify to obtain:
$$\frac{V \cdot s}{A} \cdot \frac{C}{V} = \frac{s}{\frac{C}{S}} \cdot C = \fbox{$s^2$}$$
(b) First substitute the SI units of inductance divided by resistance then simplify:
$$\frac{\frac{V \cdot s}{A}}{\Omega} = \frac{\frac{V \cdot s}{A}}{\frac{V}{A}} = \fbox{s}$$

\paragraph{}
9 - Imagine the rotation rate is increased in the ac circuit in figure 29-29. Then the rms current (a) increases, (b) does not change, (c) may increase or decrease depending on the magnitude of the original frequency, (d) may increase or decrease depending on the magnitude of the resistance, (e) decreases.\\
The rms current through the resistor $I_{rms}$ is directly proportional to $\omega$ as shown here:
$$I_{rms} = \frac{\mathcal{E}_{rms}}{R} = \frac{\mathcal{E}_{peak}}{\sqrt{2}} = \frac{NBA}{\sqrt{2}}\omega$$
therefore, \fbox{(a)} is correct.

\paragraph{}
10 - Suppose the inductance is tripled in a circuit consisting solely of a variable inductor and a variable capacitor, how you have to change the capacitance so that the natural frequency of the circuit is unchanged? (a) triple the capacitance, (b) decrease the capacitance to one-third of its original value, (c) you should not change the capacitance, (d) you cannot determine how to change the capacitance from the data given.\\
The natural frequency of an $LC$ circuit is given by
$$f_0 = \frac{1}{2\pi\sqrt{LC}}$$
We now express the natural frequencies of the circuit before and after the inductance is tripled:
$$f_0 = \frac{1}{2\pi\sqrt{LC}} \text{ and } f_0^\prime = \frac{1}{2\pi\sqrt{L^\prime C^\prime}}$$
We divide the second equation by the first and simplify:
$$\frac{f_0^\prime}{f_0} = \frac{\frac{1}{2\pi\sqrt{L^\prime C^\prime}}}{\frac{1}{2\pi\sqrt{LC}}} = \sqrt{\frac{LC}{L^\prime C^\prime}}$$
We take into account that the natural frequency is unchanged:
$$1 = \sqrt{\frac{LC}{L^\prime C^\prime}} \Rightarrow \frac{LC}{L^\prime C^\prime} = 1 \Rightarrow C^\prime = \frac{L}{L^\prime}C$$
Therefore when the inductance is tripled:
$$C^\prime = \frac{L}{3L}C = \frac{1}{3}C$$
Hence \fbox{(b)} is the correct answer.

\paragraph{}
11 - In a circuit consitig only of an ideal inductor and ideal capacitor, how does the maximum energy stored in the capacitor compare to the maximum value stored in the inductor? (a) They are the same and each equal to the total energy stored in the circuit. (b) They are the same and each equal to half of the total energy stored in the circuit. (c) The maximum energy stored in the capacitor is larger than the maximum energy stored in the inductor. (d) The maximum energy stored in the inductor is larger than the maximum energy stored in the capacitor. (e) You cannot compare the maximum energies based on the data given because the ratio of the maximum energies depends on the actual capacitance and inductance values.
The maximum energy stored in the electric field ofthe capacitor is given by
$$U_e = \frac{Q^2}{2C}$$
and the maximum energy stored in the magnetic field of the inductor is given by
$$U_m = \frac{LI^2}{2}.$$
That is because energy is conserved in a LC circuit and oscillates between the inductor and the capacitor,
$$U_c = U_m = U_{total},$$
therefore answer \fbox{(a)} is correct.

\paragraph{}
12 - Check wheter the propositions are true or false.\\
(a) A driven series $RlC$ circuit that has a high $Q$ factor has a narrow resonance curve.\\
(b) A circuit consists of an inductor, a resistor and a capacitor, all connected in series. If the resistance of the resistor is doubled, the natural frequency of the circuit remains the same.\\
(c) At resonance, the impedance of a driven series $RLC$ combination equals the resistance $R$.\\
(d) At resonance, the current in a driven series $RLC$ circuit is in phase with the voltage applied tho the combination.\\
\\
(a) True. The $Q$ factor and the width of the resonance curve at half power are related according to
$$Q = \frac{\omega_0}{\Delta \omega}$$
this means that they are inversely proportional to each other.\\
(b) True. Circuit's natural frequency depends only on the inductance $L$ of the inductor and the capacitance $C$ of the capacitor and is given by
$$\omega = \frac{1}{\sqrt{LC}}.$$
(c) True. The impedance of and $RLC$ circuit is given by
$$Z = \sqrt{R^2 + (X_L - X_C)^2}.$$
At resonance
$$X_L = X_C \text{ and so } Z = R.$$
(d) True. The phase angle $\delta$ is related to $X_L$ and $X_C$ according to
$$\delta = \tan^{-1} \left( \frac{X_L - X_C}{R} \right)$$
At resonance $X_L = X_C$ and so $\delta = 0.$

\paragraph{}
13 - Check wether propositions are true or false.\\
(a) The power factor of a driven series RLC circuit is close to zero when near resonance.\\
(b) A driven series $RLC$ circuit's power factor doesn't depend on the value of the resistance.\\
(c) A driven series $RLC$ circuit's resonance frequency doesn't depend on the value of the resistance.\\
(d) The peak current of a driven series $RLC$ circuit doesn't depend on the capacitance or the inductance when at resonance.\\
(e) For frequencies below the resonant frequency, the capacitive reactance of a driven series $RLC$ circuit is larger than the inductive reactance.\\
(f) For frequencies below the resonant frequency of a driven series $RLC$ circuit, the phase of the current leads ahead the phase of the applied voltage.\\
\\
(a) False. The power factor given by
$$\cos \delta = \frac{R}{\sqrt{(X_L - X_C)^2 +R^2}},$$
is close to 1.\\
(b) False. The power factor is given by
$$\cos \delta = \frac{R}{\sqrt{(X_L - X_C)^2 +R^2}}.$$
(c) True. The resonance frequency for a driven series $RLC$ circuit depends only on $L$ and $C$ is given by
$$\omega_{res} = \frac{1}{\sqrt{LC}}$$
(d) True. At resonance $X_L - X_C = 0$ and so $Z = R$ and the peak current is given by $I_{peak} = V_{app,peak}/R.$\\
(e) True. The capacitive reactance varies inversely with the driving frequency nd the inductive reactance varies directly with the driving frequency, hance at frequencies well below the resonance frequency the capacitive reactance is larger than the inductive reactance.\\
(f) True. For frequencies below the resonant frequency, the circuit is more capacitive than inductive and the phase constant $phi$ is negative. This means that the current leads the applied voltage.\\

\paragraph{}
14 - How can two different radio stations be heard at the same specific frequency when a receiver is tuned it, this situation often occurs while driving between two cities?\\
The power curves received by the radio have widht, hence the two frequencies coming from the radio stations can overlap as a result and you can receive signals from both stations.

\paragraph{}
15 - Check wether is true or false.\\
(a) The power factor is close to zero at frequencies much higher than or much lower than the resonant frequency of a driven series $RLC$ circuit.\\
(b) The larger the resonance width of a driven series $RLC$ circuit is, the larger the $Q$ factor for the circuit becomes.\\
(c) The larger the resistance of a driven series $RLC$ circuit is, the larger the resonance width for the circuit is.\\
\\
(a) True. The power factor is given by
$$\cos \delta = \frac{R}{\sqrt{\left( \omega L - \frac{1}{\omega C}\right)^2 + R^2}},$$
therefore for values of $\omega$ that are much higher or much lower than the resonant frequency, the term in parentheses becomes very large and $\cos \delta$ approaches to zero.\\
(b) False. When the resonance curve is reasonably narrow, the $Q$ factor can be approximated by
$$Q = \omega_0 / \Delta \omega.$$
Hence a large value for $Q$ corresponds to a narrow resonance curve.\\
(c) True. See figure 29-21

\paragraph{}
16 - Suppose an ideal transformer has $N_1$ turns on its primary and $N_2$ turn on its secondary. The average power delivered to a load resistance $R$ connected across the secondary is $P_2$ whilst the primary rms voltage is $V_1$. The rms current in the primary windings can then be expressed as (a) $P_2 / V_1,$ (b) $(N_1 / N_2)(P_2 / V_1),$ (c) $(N_2 / N_1)(P_2 / V_1),$ (d) $(N_2 / N_1)2(P_2 / V_1).$\\
Subscript 1 and 2 correspond to primary and secondary respectively. We assume no loss of power in the transformer, we can equate the power in the primary circuit to the power in the secondary circuit and solve for the rms current in th primary windings.\\
Assuming no power loss in the transformer:
$$P_1 = P_2$$
Substituting for $P$ we obtain:
$$I_{1,rms}V_{1,rms} = I_{2,rms}V_{2,rms}$$
We solve for $I$ and simplify:
$$I_{1,rms} = \frac{I_{2,rms}V_{2,rms}}{V_{1,rms}} = \frac{P_2}{V_{1,rms}}$$
hence answer $\fbox{(a)}$ is correct.

\paragraph{}
17 - Check wether propositions aret true or false:
(a) Transformers are used to change frequency.\\
(b) Transformers are used to change voltage.\\
(c) If a transformer steps up the current, it must step down the voltage.\\
(d) A step-up transformer, steps down the current.\\
(e) The standard household wall-outlet voltage in Europe is 220 V, about twice that used in the United States. If a European traveler wants her hair dryer to work properly in the United States, she should use a transformer that has more windings in its secondary coil than in its primary coil.\\
(f) The standard household wall-outlet voltage in Europe is 220 V, about twice that used in the United States. If an American traveler wants his electric razor to work properly in Europe, he should use a transformer that steps up the current.\\
\\
(a) False. A transformer is a device used to raise or lower the voltage in a circuit.\\
(b) True. A transformer is a device used to raise or lower the voltage in a circuit.\\
(c) True. If energy is to be conserved, the product of the current and voltage must be constant.\\
(d) True. Because the product of current and voltage in the primary and secondary circuits is the same, increasing the current in the secondary results in a lowering (or stepping down) of the voltage.\\
(e) True. Because electrical energy is provided at a higher voltage in Europe, the visitor would want to step-up the voltage in order to make her hair dryer work properly.\\
(f) True. Because electrical energy is provided at a higher voltage in Europe, the visitor would want to step-up the current (and decrease the voltage) in order to make his razor work properly. 

\subsection{Estimation and Approximation}
\paragraph{}
18 - Resistance and inductive reactance are included in the impedances of motors, electromagnets and transformers. Suppose that phase of the current to a large industrial plant lags the phase of the applied voltage by $25^{\circ}$ when the plant on full operation using 2.3 MW of power.  The power is supplied to the plant from a substation 4.5 km from the plant; the 60 Hz rms line voltage at the plant is 40 kV. The resistance of the transmission line from the substation to the
plant is 5.2 $\Omega$. The cost per kilowatt-hour to the company that owns the plant is \$0.14, and the plant pays only for the actual energy used. (a) Estimate the resistance and inductive reactance of the plant’s total load. (b) Estimate the rms current in the power lines and the rms voltage at the substation. (c) How much power is lost in transmission? (d) Suppose that the phase that the current lags the phase of the applied voltage is reduced to 18º by adding a bank of capacitors in series with the load. How much money would be saved by the electric utility during one month of operation, assuming the plant operates at full capacity for 16 h each day? (e) What must be the capacitance of this bank of capacitors to achieve this change in phase angle?\\
We can find the resistance and inductive reactance of the plant’s total load from the impedance of the load and the phase constant. The current in the power lines can be found from the total impedance of the load the potential difference across it and the rms voltage at the substation by applying Kirchhoff’s loop rule to the substation-transmission wires-load circuit. The power lost in transmission can be found from $P_{trans} = I_{rms}^2 R_{trans}$. We can find the cost savings by finding the difference in the power lost in transmission when the phase angle is reduced to $18^{\circ}$. Finally, we can find the capacitance that is required to reduce the phase angle to $18^{\circ}$ by first finding the capacitive reactance using the definition of $\tan \delta$ and then applying the definition of capacitive reactance to find C.\\
\\
(a) First we relate the resistance and inductive reactance of the plant's total load to Z and $\delta$:
$$R = Z \cos \delta \text{ and } X_L = Z \sin \delta$$
We then express Z in terms of the rms current $I_{rms}$ in the power lines and the rms voltage $\mathcal{E}_{rms}$ at the plant:
$$Z = \frac{\mathcal{E}_{rms}}{I_{rms}}$$
after express teh power delivered to the plant in terms of $\mathcal{E}_{rms}, I_{rms} \text{ and } \delta$ and solve for $I_{rms}:$
$$P_{av} = \mathcal{E}_{rms}I_{rms}\cos \delta \text{ and } I_{rms} = \frac{P_{av}}{\mathcal{E}_{rms}\cos \delta}$$
then substitute to obtain:
$$Z = \frac{\mathcal{E}_{rms}^2 \cos\delta}{P_{av}}$$
Afterwards substitute numerical values and evaluate Z:
$$Z = \frac{(40 kV)^2 \cos 25^{\circ}}{2.3MW} = 630 \Omega$$
Substitute numerical values and evaluate for R and $X_L:$
$$R = (630 \Omega)\cos 25^{\circ} = 571\Omega = \fbox{$0.57k\Omega$} \text{ and } X_L = (630\Omega)\sin 25^{\circ} = 266\Omega = \fbox{$0.27k\Omega$}$$
\\
(b) Find the current in the power lines:
$$I_{rms} = \frac{2.3MW}{(40kV)\cos 25^{\circ}} = 63.4 A = \fbox{63A}$$
Then apply Kirchhoff's loop rule to the circuit:
$$\mathcal{E}_{sub} - I_{rms}R_{trans} - I_{rms}Z_{tot} = 0$$
Solve for $\mathcal{E}_{sub}:$
$$\mathcal{E}_{sub} = I_{rms}(R_{trans} + Z_{tot})$$
Substitute numerical values and evaluate $\mathcal{E}_{sub}:$
$$\mathcal{E}_{sub} = (63.4A)(5.2\Omega + 630\Omega) = \fbox{40.3kV}$$
(c) The power lost in transmission is:
$$P_{trans} = I_{trans}^2 R_{trans} = (63.4A)^2(5.2\Omega) = 20.9kW = \fbox{21kW}$$
(d) Express the cost savings $\Delta C$ in terms of the difference in energy consumption $P_{25^{\circ}} - P_{18^{\circ}}\Delta t$ and the per-unit cost $u$ of the energy:
$$\Delta C = (P_{25^{\circ}} - P_{18^{\circ}})\Delta tu$$
Express the power lost in transmission when $\delta = 18^{\circ}:$
$$P_{18^{\circ}} = I_{18^{\circ}}^2R_{trans}$$
Then find the current in the transmission lines when $\delta = 18^{\circ}:$
$$I_{18^{\circ}} = \frac{2.3MW}{(40kV)\cos 18^{\circ}} = 60.5A$$
Evaluate $P_{18^{\circ}}:$
$$P_{18^{\circ}} = (60.5A)^2(5.2\Omega) = 19 kW$$
Substitute numerical values and evaluate $\Delta C:$
$$\Delta C = (20.9kW - 19kW)(16 h/d)\left( 30 \frac{d}{\text{month}}\right) \left( \frac{\$0.14}{kW \cdot h}\right) = \fbox{\$128}$$
(e) The required capacitance is given by:
$$C = \frac{1}{2\pi fX_C}$$
We then relate the new phase $\delta$ to the inductive reactance $X_L,$ the reactance due to the added capacitance $X_C,$ and the resistance of the load R:
$$\tan \delta = \frac{X_L - X_C}{R} \Rightarrow X_C = X_L - R \tan \delta$$
We substitute for $X_C:$
$$C = \frac{1}{2\pi f(X_L R \tan \delta)}$$
We then substitute numerical values and evaluate C:
$$C = \frac{1}{2\pi(60s^{-1})(266\Omega - (571\Omega)\tan 18^{\circ})} = \fbox{$33\mu F$}$$

\section{Alternating Current in Resistors, Inductors, and Capacitors}
\paragraph{}
19 - Suppose we have a 100-W light bulb in a standard 120-V-rms socket. Find (a) the rms current, (b) the peak current, and (c) the peak power.\\
\\
We use $P_{av} = \mathcal{E}_{rms}I_{rms}$ to find $I_{rms},$ $I_{peak} = \sqrt{2}I_{rms}$ to find $I_{peak},$ and $P_{peak} = I_{peak} \mathcal{E}_{peak}$ to find $P_{peak}.$\\
(a) We find the relationship of average power delivered by the source to the rms
voltage across the bulb and the rms current through it:
$$P_{av} = \mathcal{E}_{rms}I_{rms} \Rightarrow I_{rms} = \frac{P_{av}}{\mathcal{E}_{rms}}$$
We then substitute numerical values and evaluate $I_{rms}:$
$$I_{rms} = \frac{100W}{120V} = 0.8333A = \fbox{0.833A}$$
(b) We do express $I_{peak}$ in terms of $I_{rms}:$
$$I_{peak} = \sqrt{2}I_{rms}$$
substitute for $I_{rms}$ and evaluate $I_{peak}:$
$$I_{peak} = \sqrt{2}(0.8333A) = 1.1785A = \fbox{1.18A}$$
(c) Afterwards we need to express the maximum power in terms of the maximum voltage and maximum current, and substitute numerical values evaluating $P_{peak}:$
$$P_{peak} = I_{peak} \mathcal{E}_{peak}$$
$$P_{peak} = (1.1785A)\sqrt{2}(120V) = \fbox{200W}$$

\paragraph{}
20 - Suppose we have a circuit breaker for a current of 15 A rms at a voltage of 120 V rms. (a) What is the largest peak current that the breaker can carry? (b) What is the maximum average power that can be supplied by this circuit?\\
\\
We can use $I_{peak} = \sqrt{2}I_{rms}$ to find the largest peak current that the breaker can carry and $P_{av} = I_{rms}V_{rms}$ to find the average power supplied by the circuit.\\
(a) We do express $I_{peak}$ in terms of $I_{rms}$ and relate the average power to the rms current and voltage:
$$I_{peak} = \sqrt{2}I_{rms} = \sqrt{2}(15A) = \fbox{21A}$$
$$P_{av} = I_{rms}V_{rms} = (15A)(120V) = \fbox{1.8kW}$$

\paragraph{}
21 - What is the reactance of a 1.00mH inductor at (a) 60 Hz, (b) 600 Hz, and (c) 6.00 kHz?\\
\\
Utilize $X_L = \omega L$ to determine the reactance of the inductor at any frequency. We do so by expressing the inductive reactance as a function of $f:$
$$X_L = \omega L = 2\pi fL$$
(a) At $f = 60Hz:$
$$X_L = 2\pi (60s^{-1})(1mH) = \fbox{$0.38\Omega$}$$
(b) At $f = 600Hz:$
$$X_L = 2\pi (600s^{-1})(1mH) = \fbox{$3.77\Omega$}$$
(c) At $f = 6kHz:$
$$X_L = 2\pi (6ks^{-1})(1mH) = \fbox{$37.7\Omega$}$$

\paragraph{}
22 - An inductor has a reactance of 100 $\Omega$ at 80 Hz. (a) What is its inductance? (b) What is its reactance at 160 Hz?\\
\\
We can use $X_L = \omega L$ to find the inductance of the inductor at any frequency.\\
(a) We do relate the reactance or the inductor the its inductance and solve for $L$ doing its evaluation:
$$X_L = \omega L = 2\pi fL \Rightarrow L = \frac{X_L}{2\pi f}$$
$$L = \frac{100 \Omega}{2\pi (80s^{-1})} = 0.199H = \fbox{0.20 H}$$
(b) At 160 Hz:
$$X_L = 2\pi (160s^{-1})(0.199H) = \fbox{$0.20k\Omega$}$$

\paragraph{}
23 - Consider a $10\mu F$ capacitor, at what frequency would its reactance equal a reactance of an 10mH inductor?\\
\\
If we equate reactances of the capacitor and inductor we then can solve for the frequency.\\
We do express the reactance of the inductor, then we express the reactance of the capacitor equating these reactances:
$$X_L = \omega L = 2\pi fL$$
$$X_C = \frac{1}{\omega C} = \frac{1}{2\pi fC}$$
$$2\pi fL = \frac{1}{2\pi fC} \Rightarrow f = \frac{1}{2\pi} \sqrt{\frac{1}{LC}}$$
We finally substitute numerical values and evaluate f:
$$f = \frac{1}{2\pi}\sqrt{\frac{1}{(10\mu F)(1mH)}} = \fbox{1.6kHz}$$

\paragraph{}
24 - What is the reactance of a 1.00nF capacitor at (a) 60.0 Hz, (b) 6.00 kHz, and (c) 6.00 MHz?\\
\\
We can use $X_C = 1 / \omega C$ to find the reactance of the capacitor at any frequency. By expressing the capacitive reactance as a function of $f:$
$$X_C = \frac{1}{\omega C} = \frac{1}{2\pi fC}$$
(a) At $f = 60 Hz$:
$$X_C = \frac{1}{2\pi(60s^{-1})(1nF)} = \fbox{$2.65M\Omega$}$$
(b) At $f = 6.00 kHz$:
$$X_C = \frac{1}{2\pi(6.00kHz)(1nF)} = \fbox{$26.5k\Omega$}$$
(c) At $f = 6.00 MHz$:
$$X_C = \frac{1}{2\pi(6.00MHz)(1nF)} = \fbox{$26.5\Omega$}$$

\paragraph{}
 25 - A 20-Hz ac generator that produces a peak emf of 10 V is connected to a $20-\mu F$ capacitor. Find (a) the peak current and (b) the rms current.\\
\\
We can use $I_{peak} = \mathcal{E}_{peak} / X_C$ and $X_C = 1 / \omega C$ to express $I_{peak}$ as a function of $\mathcal{E}_{peak}, f$ and C. Once we evaluate $I_{peak},$ we can use $I_{rms} = I_{peak} / \sqrt{2}$ to find $I_{rms}$.\\
We begin by expressing $I_{peak}$ in terms of $\mathcal{E}_{peak}$ and $X_C$$$I_{peak} = \frac{\mathcal{E}_{peak}}{X_C}$$
We then express the capacitive reactance and subsitute for $X_C$ and simplify:
$$I_{peak} = 2\pi fC\mathcal{E}_{peak}$$
(a) Substitute numerical values and evaluate $I_{peak}:$
$$I_{peak} = 2\pi(20s^{}-1)(20\mu F)(10V) = 25.1mA = \fbox{25mA}$$
(b) Express $I_{rms} = \frac{I_{peak}}{\sqrt{2}} = \frac{25.1mA}{\sqrt{2}} = 18mA$

\paragraph{}
26 - At what frequency is the reactance of a 10-$\mu F$ capacitor (a) 1.00 $\Omega$, (b) 100 $\Omega$, and (c) 10.0 $m\Omega$?\\
\\
We can use $X_C = 1 / \omega C = 1 / 2\pi fC$ to relate the reactance of the capacitor to the frequency:
$$X_C = \frac{1}{\omega C} = \frac{1}{2\pi fC} \Rightarrow f = \frac{1}{2\pi CX_C}$$
(a) Find $f$ when $X_C = 1.00 \Omega$:
$$f = \frac{1}{2\pi(10\mu F)(1.00\Omega)} = \fbox{16kHz}$$
(b) Find $f$ when $X_C = 100 \Omega$:
$$f = \frac{1}{2\pi(10\mu F)(100\Omega)} = \fbox{0.16kHz}$$
(c) Find $f$ when $X_C = 10.0m \Omega$:
$$f = \frac{1}{2\pi(10\mu F)(10.0m \Omega)} = \fbox{1.6MHz}$$

\paragraph{}
27 - Suppose a circuit consists of two ideal ac generators and a 25-$\Omega$ resistor, all connected in series. The potential difference across the terminals of one of the generators is given by $V_1 = (5.0 V) \cos( \omega t – \alpha)$, and the potential difference across the terminals of the other generator is given by $V_2 = (5.0 V) \cos( \omega t + \alpha )$, where $\alpha = \pi / 6$. (a) Use Kirchhoff’s loop rule and a trigonometric identity to find the peak current in the circuit. (b) Use a phasor diagram to find the peak current in the circuit. (c) Find the current in the resistor if $\alpha = \pi / 4$ and the amplitude of $V_2$ is increased from 5.0 V to 7.0 V.\\
\\
We can find the sum of the phasors $V_1$ and $V_2$ using this trigonometric identity
$$\cos \theta + \cos \phi = 2 \cos \frac{1}{2}(\theta + \phi) \cos \frac{1}{2}(\theta - \phi)$$
then we use this sum to express $I$ as a function of time. In (b) we use a phasor diagram to obtain the same result and in (c) we use the phasor diagram appropriate to the given voltages to express the current as a function of time.\\
(a) Apply the Kirchhoff's loop rule to the circuit and solve for I:
$$V_1 + V_2 - IR = 0$$
$$I = \frac{V_1 + V_2}{R}$$
Then we use the trigonometric identity described above to find $V_1 + V_2:$
$$V_1 + V_2 = (5.0V)[\cos (\omega t - \alpha ) + \cos (\omega t + \alpha)] = (5V)[2\cos \frac{1}{2}(2\omega t)\cos \frac{1}{2} (-2\alpha)] =$$
$$(10V)\cos \frac{\pi}{6}\cos \omega t = (8.66V)\cos \omega t$$
We substitute for $V_1 + V_2$ and $R$ to obtain:
$$I = \frac{(8.66V)\cos \omega t}{25\Omega} = (0.346A)\cos \omega t = (0.35A)\cos \omega t$$
$$I_{peak} = \fbox{0.35A}$$
(b) We express the magnitude of the current in $R:$
$$\lvert I \rvert = \frac{\lvert \vec{V} \rvert}{R}$$
The phasor diagram for the voltages is shown. We can find $\vec V$ using vector addition:
$$\lvert \vec V \rvert = 2 \lvert \vec V_1 \rvert cos 30^{\circ} = 2(5.0V)\cos 30^{\circ} = 8.66V$$
We then substitute for $\lvert \vec V \rvert$ and $R$ to obtain: $\lvert I \rvert = \frac{8.66V}{25\Omega} = 0.346A$ and $I = (0.35A) \cos \omega t$ where $I_{peak} = \fbox{0.35A}$
(c) The phasor diagram is shown. Note that the phase angle between $V_1$ and $V_2$ is now $90^{\circ}$. We use the Pythagorean theorem to find $\lvert \vec V \rvert$:
$$\lvert \vec V \rvert = \sqrt{\lvert \vec V_1 \rvert ^2 + \lvert \vec V_2 \rvert ^2} = \sqrt{(5.0V)^2 + (7.0V)^2} = 8.60V$$
Then we express $I$ as a function of $t$:
$$I = \frac{\lvert \vec V \rvert}{R}\cos (\omega t + \delta)$$
where $\delta = 45^{\circ} - (90^{\circ} - \alpha) = \alpha - 45^{\circ} = \tan ^{-1} \frac{7.0V}{5.0V} - 45^{\circ} = 9.462^{\circ} = 0.165rad$\\
Finally substitute numerical values and evaluate $I$:
$$I = \frac{8.60V}{25\Omega}\cos (\omega t + 0.165rad) = \fbox{$(0.34A)\cos (\omega t + 0.17rad)$}$$

\section{Undriven Circuits Containing Capacitors, Resistors and Inductors}
\paragraph{}
28 - (a) Show that 1 LC has units of inverse seconds by substituting SI
units for inductance and capacitance into the expression. (b) Show that $\omega$ 0 L/R (the expression for the Q-factor) is dimensionless by substituting SI units for angular frequency, inductance, and resistance into the expression.
\\
We begin by substituting the units of the various physical quantities in $1 / \sqrt{LC}$ and $Q = \omega{_0}L / R$ to establish their units.\\
(a) We do substitute the units for $\omega{_0}, L$ and $C$ in the expression $1 / \sqrt{LC}$ and then simplify:
$$\frac{1}{\sqrt{H\cdot F}} = \frac{1}{\sqrt{(\Omega \cdot s)(\frac{s}{\Omega})}} = \frac{1}{\sqrt{s^2}} = \fbox{$s^{-1}$}$$ 
(b) We do substitute the units for $\omega{_0}, L$ and $R$ in the expression $Q = \omega{_0}L / R$ and simplify:
$$\frac{\frac{1}{s} \cdot \frac{V \cdot s}{A}}{\frac{V}{A}} = 1 \Rightarrow \fbox{unitless}$$

\paragraph{}
29 - (a) What is the period of oscillation of an LC circuit consisting of an ideal 2.0-mH inductor and a 20- $\mu$ F capacitor? (b) A circuit that oscillates consists solely of an 80- $\mu$ F capacitor and a variable ideal inductor. What inductance is needed in order to tune this circuit to oscillate at 60 Hz?\\
$T = 2\pi / \omega$ and $\omega = 1 / \sqrt{LC}$ can be used to relate $T, f$ to $L$ and $C$.\\
(a) We express the period of oscillation:
$$T = \frac{2\pi}{\omega}$$
Taking into account that for a $LC$ circuit $\omega = 1 / \sqrt{LC}$, we substitute for $\omega$ obtaining:
$$T = 2\pi \sqrt{LC} = 2\pi \sqrt{(2.0mH)(20\mu F)} = \fbox{1.3ms}$$
From the equation for $\omega$ we solve for $L$:
$$L = \frac{T^2}{4\pi^2C} = \frac{1}{4\pi^2 f^2 C} = \frac{1}{4\pi^2 (60^{-1})(80\mu F)} = \fbox{88mH}$$

\paragraph{}
30 - An LC circuit has capacitance C 0 and inductance L. A second LC circuit has capacitance 12 C 0 and inductance 2L, and a third LC circuit has capacitance 2C 0 and inductance 12 L. (a) Show that each circuit oscillates with the same frequency. (b) In which circuit would the peak current be greatest if the peak voltage across the capacitor in each circuit was the same?\\
\\
The expression $f_0 = 1 / 2\pi \sqrt{LC}$ can be used for the resonance frequency of a $LC$ circuit showing that each circuit oscillates with the same frequency. For (b) we can use $I_{peak} = \omega Q_0,$ where $Q_0$ is the charge of the capacitor at time zero, and the definition of capacitance $Q_0 = CV$ to express $I_{peak}$ in terms of $\omega, C, V$.\\
We do express the resonance frequency for a $LC$ circuit 
$$f_0 = \frac{1}{2\pi \sqrt{LC}}$$
(a) We separate the product of $L$ and $C$ for each circuit:
$$\text{Circuit 1:} L_1C_1 = L_1C_0,$$
$$\text{Circuit 2:} L_2C_2 = (2L_1)(\frac{1}{2}C_0) = L_1C_1,$$
$$\text{Circuit 3:} L_3C_3 = (\frac{1}{2}L_1)(2C_0) = L_1C_1,$$
$$L_1C_1 = L_2C_2 = L_3C_3$$
Therefore, the resonance frequencies of the three circuits are equal.\\
(b) We do express the I peak in terms of the charge stored in the capacitor and $Q_0$ in terms of the capacitance of the capacitor and the potential difference across the capacitor, then we substitute:
$$I_{peak} = \omega Q_0, \quad Q_0 = CV \Rightarrow I_{peak} = \omega CV$$
We have I peak directly proportional to C when $\omega$ and V are held constant. Hence the circuit with a capacitance of $2C_0$ has the greatest peak current.

\paragraph{}
31 - A 5.0- $\mu$ F capacitor is charged to 30 V and is then connected across an ideal 10-mH inductor. (a) How much energy is stored in the system? (b) What is the frequency of oscillation of the circuit? (c) What is the peak current in the circuit?\\
\\
The energy stored in the electric field of the capacitor can be found using $U = \frac{1}{2}CV^2$, the frecuency can be found by these relations $\omega{_0} = 2\pi f_0 = 1 / \sqrt{LC}$, and I peak can be determined by using these relations $I_{peak} = \omega Q_0$ and $Q_0 = CV$.\\
(a) We express the energy stored in the system as function of C and V, substituting its numerical values:
$$U = \frac{1}{2}CV^2 = \frac{1}{2}(5.0\mu F)(30V)^2 = \fbox{2.3mJ}$$
(b) We express the resonance frequency in terms of L and C, and substitute numerical values:
$$\omega{_0} = 2\pi f_0 = \frac{1}{\sqrt{LC}} \Rightarrow f_0 = \frac{1}{2\pi \sqrt{LC}} = \frac{1}{2\pi\sqrt{(10mH)(5.0\mu F)}} = \fbox{0.71kHz}$$
(c) I peak can be expressed in terms of the charge stored in the capacitor and $Q_0$ can be expressed in terms of the capacitance of the capacitor and the potential difference accross the capacitor, using these two, we substitute for $Q_0$:
$$I_{peak} = \omega Q_0, \quad Q_0 = CV \Rightarrow I_{peak} = \omega CV$$
$$I_{peak} = 2\pi (712s^{-1})(5.0 \mu F)(30V) = \fbox{0.67A}$$

\paragraph{}
32 - A coil with internal resistance can be modeled as a resistor and an ideal inductor in series. Assume that the coil has an internal resistance of 1.00 $\Omega$ and an inductance of 400 mH. A 2.00- $\mu$ F capacitor is charged to 24.0 V and is then connected across coil. (a) What is the initial voltage across the coil? (b) How much energy is dissipated in the circuit before the oscillations die out? (c) What is the frequency of oscillation the circuit? (Assume the internal resistance is sufficiently small that has no impact on the frequency of the circuit.) (d) What is the quality factor of the circuit?\\
\\
(a) To find the initial voltage across the coil we can apply the Kirchhoff's loop rule. (b) The total energy lost by the Joule heating is the total energy initially stored in the capacitor. (c) The natural frequency of the circuit is given by $f_0 = 1 / 2\pi \sqrt{LC}$. (d) The quality factor can be found by its definition.\\
(a) Applying Kirchhoff's loop rule gives that initial voltage across the coil is \fbox{24.0V}\\
(b) All the energy initially stored in the capacitor will be dissipated as Joule heat in the resistor because the ideal inductor can not dissipate energy as heat:
$$U = \frac{1}{2}CV^2 = \frac{1}{2}(2.00\mu F)(24.0V)^2 = \fbox{0.576mJ}$$
(c) We find the natural frequency of the circuit:
$$f_0 = \frac{1}{2\pi \sqrt{LC}} = \frac{1}{2\pi \sqrt{(400mH)(2.00\mu F)}} = \fbox{178Hz}$$
(d) The quality factor definition is $Q = \omega{_0} L / R$, we can substitute for $\omega{_0}$ and simpify, finally we substitute numerical values:
$$Q = \frac{\frac{1}{\sqrt{LC}}L}{R} = \frac{1}{R}\sqrt{\frac{L}{C}} = \frac{1}{1.00\Omega}\sqrt{\frac{400mH}{2.00\mu F}} = \fbox{447}$$

\paragraph{}
33 - An inductor and a capacitor are connected, as shown in Figure 29-30. Initially, the switch is open, the left plate of the capacitor has charge Q 0 . The switch is then closed. (a) Plot both Q versus t and I versus t on the same graph, and explain how it can be seen from these two plots that the current leads the charge by 90o. (b) The expressions for the charge and for the current are given by Equations 29-38 and 29-39, respectively. Use trigonometry and algebra to show that the current leads the charge by 90o.\\
\\
We can obtain a differential equation for the circuit by applying Kirchhoff's loop rule, letting Q represent the instantaneous charge of the capacitor. Solving this equation we obtain an expression for the charge on the capacitor as a function of time and by differentiating this expression with respect to time, and expression for the current as a function of time.\\
We apply Kirchhoff's loop rule to a clockwise loop just after the switch is closed:
$$\frac{Q}{C} + L\frac{dI}{dt} = 0$$
Substitute $I = dQ/dt$:
$$L\frac{d^2Q}{dt^2} + \frac{Q}{C} = 0, \text{ hence } \frac{d^2Q}{dt^2} + \frac{1}{LC}Q = 0$$
The solution to this differential equation is: $Q(t) = Q_0\cos (\omega t - \delta)$ where $\omega = \sqrt{1 / LC}$. Because $Q(0) = Q_0,$ and $\delta = 0$ and $Q(t) = Q_0\cos \omega t$. The current in the circuit is the derivative of Q with respect to t:
$$I = \frac{dQ}{dt} = \frac{d}{dt}[Q_0 \cos \omega t] = -\omega Q_0 \sin \omega t$$
(a) A spreadsheet program was used to plot the following graph showing both the charge on the capacitor and the current in the circuit as functions of time. L, C, and Q 0 were all arbitrarily set equal to one to obtain these graphs. Note that the current leads the charge by one-fourth of a cycle or 90 degrees.\\
(b) The equation for the current is: $I = -\omega Q_0 \sin \omega t$, we can use the identity between sine and cosine functions: $-sin \theta = \cos \left( \theta + \frac{\pi}{2}\right)$. Using this identity we can rewrite the equation for the current:
$$I = -\omega Q_0 \sin \omega t = \fbox{$\omega Q_0 \cos \left( \omega t + \frac{\pi}{2}\right)$}$$
Therefore, the current leads the charge by 90 degrees.

\section{Driven RL Circuits}
\paragraph{}
34 - A circuit consists of a resistor, an ideal 1.4-H inductor and an ideal 60-Hz generator, all connected in series. The rms voltage across the resistor is 30 V and the rms voltage across the inductor is 40 V. (a) What is the resistance of the resistor? (b) What is the peak emf of the generator?\\
\\
The ratio $V_R$ to $V_L$ can be used to find the resistance of the circuit. For (b) we can use the fact that in a LR circuit, $V_L$ leads $V_R$ by 90 degrees to find the ac input voltage.\\
(a) We do express the potential differences accross R and L in terms of the common current through these components and express R in thee terms.
$$V_L = IX_L = I\omega L, \quad V_R = IR$$
$$\frac{V_R}{V_L} = \frac{IR}{I\omega L} = \frac{R}{\omega L} \Rightarrow R = \left( \frac{V_R}{V_L} \right)\omega L$$
We substitute numerical values and evaluate R:
$$R = \left( \frac{30V}{40V} \right) 2\pi (60s^{-1})(1.4H) = \fbox{$0.40k\Omega$}$$
(b) $V_R$ leads $V_L$ by 90 degrees in a LR circuit:
$$V_{peak} = \sqrt{2}V_{rms} = \sqrt{2}\sqrt{V_R^2 + V_L^2}$$
$$V_{peak} = \sqrt{2} \sqrt{(30V)^2 + (40V)^2} = \fbox{71V}$$

\paragraph{}
35 - A coil that has a resistance of 80.0 $\Omega$ has an impedance of 200 $\Omega$ when driven at a frequency of 1.00 kHz. What is the inductance of the coil?\\
\\
The definition of $X_L$ in a $LR$ circuit can be used to find $L$.\\
From the definition of impedance of a coil in terms of its resistance and inductive reactance we can solve for $X_L$:
$$Z = \sqrt{R^2 + X_L^2} \Rightarrow X_L = \sqrt{Z^2 - R^2}$$
Remember that $X_L$ can be expressed in terms of $L$, $X_L = 2\pi fL$. If we equate these equations we can solve for L:
$$2\pi fL = \sqrt{Z^2 - R^2} \Rightarrow L = \frac{\sqrt{Z^2 - R^2}}{2\pi f}$$
Now we only need to substitute numerical values:
$$L = \frac{\sqrt{(200\Omega)^2-(80.0\Omega)^2}}{2\pi (1.00kHz)} = \fbox{29.2mH}$$

\paragraph{}
36 - A two conductor transmission line simultaneously carries a superposition of two voltage signals, so the potential difference between the two conductors is given by V = V 1 + V 2 , where V 1 = (10.0 V) cos($\omega{_1}$t) and V 2 = (10.0 V) cos( $\omega{_2}$t), where $\omega{_1}$ = 100 rad/s and $\omega{_2}$ = 10 000 rad/s. A 1.00 H inductor and a 1.00 k$\Omega$ shunt resistor are inserted into the transmission line as shown in Figure 29-31. (Assume that the output is connected to a load that draws only an insignificant amount of current.) (a) What is the voltage (V out ) at the output of the transmission line? (b) What is the ratio of the low-frequency amplitude to the high-frequency amplitude at the output?\\
\\
The two output voltage signals can be expressed as the product of the current from each source and $R = 1.00 k\Omega$. The impedance definition and given voltage signals can help us for determine the currents due to each source.\\
(a) We do express the output voltage signals in terms of the potential difference across the resistor:
$$V_{1,out} = I_1R \quad V_{2,out} = I_2R$$
We need $I_1$ and $I_2$:
$$I_1 = \frac{V_1}{Z_1} = \frac{(10.0V)\cos 100t}{\sqrt{(1.00k\Omega)^2+[(100s^{-1})(1.00H)]^2}} = (9.95mA)\cos 100t$$
$$I_2 = \frac{V_2}{Z_2} = \frac{(10.0V)\cos 10^4t}{\sqrt{(1.00k\Omega)^2+[(10^4s^{-1})(1.00H)]^2}} = (0.995mA)\cos 10^4t$$
Substituting these values in voltage expressions:
$$V_{1,out} = (1.00k\Omega)(9.95mA)\cos 100t = \fbox{$(9.95V)\cos 100t$}$$
$$V_{2,out} = (1.00k\Omega)(0.995mA)\cos 10^4t = \fbox{$(0.995V)\cos 10^4t$}$$
where $\omega_1 = 100 rad/s$ and $\omega_2 = 10000 rad/s$

\paragraph{}
37 - A coil is connected to a 120-V rms, 60-Hz line. The average power supplied to the coil is 60 W, and the rms current is 1.5 A. Find ( a ) the power factor, ( b ) the resistance of the coil, and ( c ) the inductance of the coil. ( d ) Does the current lag or lead the voltage? Explain your answer. ( e ) Support your answer to Part ( d ) by determining the phase angle.\\
\\
There is a relationship between the power factor and the average power supplied to the coil $P_{av} = \mathcal{E}_{rms}I_{rms}\cos \delta$. In (b) $P_{av} = I_{rms}^2R$ this expression gives the needed relationship for R. The resistance of the coil can be found by $X_L = \omega L = R \tan \delta$ this expression relates the resistance, phase angle and inductance. By noting if the circuit is inductive, we can decide if the current leads or lags the voltage.\\
(a) We express the average power supplied to the coil in terms of the power factor:
$$P_{av} = \mathcal{E}_{rms}I_{rms} \cos \delta \Rightarrow \cos \delta = \frac{P_{av}}{\mathcal{E}_{rms}I_{rms}}$$
Substituting numerical values:
$$\cos \delta = \frac{60W}{(120V)(1.5A)} = 0.333 = \fbox{0.33}$$
(b) We do express the power supplied by the source in terms of the resistance of the coil and substitute numerical values:
$$P_{av} = I_{rms}^2 \Rightarrow R = \frac{P_{av}}{I_{rms}^2} = \frac{60W}{(1.5A)^2} = 26.7\Omega = \fbox{$27\Omega $}$$
(c) We relate the inductive reactance to the resistance and phase angle, solving for L:
$$X_L = \omega L = R \tan \delta \Rightarrow L = \frac{R \tan \delta}{\omega} = \frac{R \tan[\cos^{-1}(0.333)]}{2\pi f}$$
$$L = \frac{(26.7\Omega)\tan (70.5^{\circ})}{2\pi (60s^{-1})} = \fbox{0.20H}$$
(d) We just need to evaluate $X_L$:
$$X_L = (26.7\Omega)\tan (70.5^{\circ}) = 75.4\Omega$$
The circuite is inductive, hence the current lags the voltage.\\
(e) From part (a):
$$\delta = \cos^{-1}(0.333) = 72^{\circ}$$

\paragraph{}
38 - A 36-mH inductor that has a resistance of 40 $\Omega$ is connected to an ideal ac voltage source whose output is given by $\epsilon$ = (345 V) cos(150 $\pi$ t ), where t is in seconds. Determine ( a ) the peak current in the circuit, ( b ) the peak and rms voltages across the inductor, ( c ) the average power dissipation, and ( d ) the peak and average magnetic energy stored in the inductor.\\
\\
(a) Use $I_{peak} = \mathcal{E}_{peak}/\sqrt{R^2 + (\omega L)^2}$ and $V_{L,peak} = I_{peak}X_L = \omega LI_{peak}$ to find the peak current in the circuit and the peak voltage across the inductor. (b) Once we've found $V_{L,peak}$ we can find $V_{L,rms}$ using $V_{L,rms} = V_{L,peak} / \sqrt{2}$. (c) We can use $P_{av} = \frac{1}{2}I^2_{rms}R$ to find the average power dissipation. (d) $U_{L,peak} = \frac{1}{2}LI_{peak}^2$ to find the peak and average magnetic energy stored in the inductor. The average energy stored in the magnetic field of the inductor can be found using $U_{L,av} = \int P_{av}dt$.\\
(a) Apply Kirchhoff's loop rule to the circuit:
$$\mathcal{E} - IZ = 0 \Rightarrow I = \frac{\mathcal{E}}{Z} = \frac{\mathcal{E}}{\sqrt{R^2 + (\omega L)^2}}$$
Substituting numerical values to find for I:
$$I_{peak} = \frac{(345V)\cos (150 \pi t)}{\sqrt{(40\Omega)^2} + [(150\pi s^{-1})(36mH)]^2} = (7.94A)\cos (150\pi t) = \fbox{7.9A}$$
(b) We can find $\mathcal{E}$ as follows:
$$\mathcal{E} = (345V) \cos (150 \pi t)$$
$$V_{L,peak} = \fbox{345V}$$
$$V_{L,rms} = \frac{V_{L,peak}}{\sqrt{2}} = \frac{345V}{\sqrt{2}} = \fbox{244V}$$
(c) We can relate the average power dissipation to $I_{peak}$ and R:
$$P_{av} = I_{rms}^2R = \left( \frac{I_{peak}}{\sqrt{2}} \right)^2R = \frac{1}{2}I_{peak}^2R$$
Substituting numerical values:
$$P_{av} = \frac{1}{2}(7.94A)^2(40\Omega) = \fbox{1.3kW}$$
(d) The maximum energy stored in the magnetic field of the inductor can be found as follows:
$$U_{L,peak} = \frac{1}{2}LI_{peak}^2 = \frac{1}{2}(36mH)(7.94A)^2 = \fbox{1.1J}$$
From the definition $U_{L,av} = \frac{1}{T}\int_0^TU(t)dt$ and $U(t) = \frac{1}{2}L[I(t)]^2$. We do substitute $U(t)$ and obtain:
$$U_{L,av} = \frac{L}{2T}\int_0^T[I(t)]^2dt$$
We evaluate the integral:
$$U_{L,av} = \frac{L}{2T}\left[ \frac{1}{2}I_{peak}^2\right]T = \frac{1}{4}LI_{peak}^2$$
Finally we do substitute numerical values and evaluate:
$$U_{L,av} = \frac{1}{4}(36mH)(7.94A)^2 = \fbox{0.57J}$$

\paragraph{}
39 - A coil that has a resistance R and an inductance L has a power factor equal to 0.866 when driven at a frequency of 60 Hz. What is the coil’s power factor it is driven at 240 Hz?\\
\\
The relationship between $X_L$ and $R$ define the power factor when the coil is driven at a frequency of 60 Hz and then from the definition of $X_L$ we can relate the inductive reactance at 240 Hz to the inductive reactance at 60 Hz, finally we can use the power factor definition to determine it at 240 Hz.\\
We use the definition of power factor to relate $R$ and $X_L$:
$$\cos \delta = \frac{R}{Z} = \frac{R}{\sqrt{R^2 + X_L^2}} \Rightarrow \cos^2 \delta = \frac{R^2}{R^2 + X_L^2}$$
We solve for 60 Hz and substitute for $\cos \delta$:
$$X_L^2(60Hz) = R^2\left( \frac{1}{\cos^2 \delta} - 1 \right) = R^2\left( \frac{1}{(0.866)^2} - 1 \right) = \frac{1}{3}R^2$$
We use the definition of $X_L$:
$$X_L^2(f) = 4\pi f^2L^2 \text{ and } X_L^2(f') = 4\pi f'^2L^2$$
Combining these two equations:
$$\frac{X_L^2(f')}{X_L^2(f)} = \frac{4\pi f'^2L^2}{4\pi f^2L^2} = \frac{f'^2}{f^2} \Rightarrow X_L^2(f') = \left( \frac{f'}{f} \right)^2X_L^2(f)$$
Finally we substitute numerical values:
$$X_L^2(240Hz) = \left( \frac{240s^{-1}}{60s^{-1}} \right)^2 X_L^2(60Hz) = 16\left( \frac{1}{3}R^2 \right) = \frac{16}{3}R^2$$

\paragraph{}
40 - A resistor and an inductor are connected in parallel across an ideal ac voltage source whose output is given by E = E peak cos omega t as shown in Figure 29-32.Show that (a) the current in the resistor is given by I R = E peak /R cos omega t, (b) the current in the inductor is given by I L = E peak /X L cos( omega t – 90), and (c) the current in the voltage source is given by I = I R + I L = I peak cos( omega t – delta ), where I peak = E max / Z.\\
\\
The power factor definition relates R and $X_L$:
$$\cos \delta = \frac{R}{Z} = \frac{R}{\sqrt{R^2 + X_L^2}} \Rightarrow \cos^2 \delta = \frac{R^2}{R^2 + X_L^2}$$
Solving for $X_L^2(60Hz)$:
$$X_L^2(60Hz) = R^2 \left( \frac{1}{\cos^2 \delta} - 1\right)$$
We substitute and simplify:
$$X_L^2(60Hz) = R^2 \left( \frac{1}{(0.866)^2} - 1\right) = \frac{1}{3}R^2$$
Using the $X_L$ definition:
$$X_L^2(f) = 4\pi f^2L^2 \text{ and } X_L^2(f') = 4\pi f'^2L^2$$
We do simplify these two equations by dividing them and solve for $X_L$:
$$\frac{X_L^2(f')}{X_L^2(f)} =  \frac{4\pi f'^2L^2}{4\pi f^2L^2} = \frac{f'^2}{f^2} \Rightarrow X_L^2(f') = \left( \frac{f'}{f} \right)^2X_L^2(f)$$
Substituting numerical values:
$$X_L^2(240Hz) = \left( \frac{240s^{-1}}{60s^{-1}} \right)^2 X_L^2(60Hz) = 16 \left( \frac{1}{3}R^2 \right) = \frac{16}{3}R^2$$
We do substitute in the first equation to obtain:
$$(\cos \delta)_{240Hz} = \frac{R}{\sqrt{R^2 + \frac{16}{3}R^2}} = \sqrt{\frac{3}{19}} = \fbox{0.397}$$

\paragraph{}
40 - A resistor and an inductor are connected in parallel across an ideal ac voltage source whose output is given by E = E peak cos omega t as shown in Figure 29-32. Show that (a) the current in the resistor is given by I R = E peak /R cos omega t, (b) the current in the inductor is given by I L = E peak /X L cos( omega t – 90), and (c) the current in the voltage source is given by I = I R + I L = I peak cos( omega t – delta ), where I peak = E max /Z.\\
\\
The voltage drops across hte inductor and the resistor are equal because they are connected in parallel. The sum of the current through the resistor and through the inductor sum up the total current. The two currents are not in phase, phasors are needed to calculate their sum. The magnitude of the phasors are equal to the amplitudes of the applied voltage and the currents, i.e. $\lvert\mathcal{\vec E} \rvert = \mathcal{E}, \lvert \vec I \rvert = I_{peak}, \lvert \vec I_R\rvert = I_{R,peak}, \lvert \vec I_L \rvert = I_{L,peak}$ \\
(a) The voltage given is supplied by the ac source $\mathcal{E} = \mathcal{E}\cos \omega t$. Therefore the voltage drop across the load resistor and the inductor is:
$$I_R = I_{R,peak}\cos \omega t$$
and as $I_{R,peak} = \frac{\mathcal{E}_{peak}}{R}$:
$$I_R = \frac{\mathcal{E}_{peak}}{R}\cos \omega t$$
(b) The current in the inductor lags the applied voltage by $90^{\circ}$:
$$I_L = I_{L,peak} \cos (\omega t - 90^{\circ})$$
and as $I_{L,peak} = \frac{\mathcal{E}_{peak}}{X_L}$:
$$I_L = \frac{\mathcal{E}_{peak}}{X_L}\cos (\omega t - 90^{\circ})$$
(c) The sum of the currents through the parallel branches sum up the net current $I$:
$$I = I_R + I_L$$
It helps if we draw the phasor diagram of the circuit. The projections of the phasors onto the horizontal axis are the instantaneous values. The current in the resistor is in phase with the applied voltage, and the current in the inductor lags the applied voltage by 90 degrees. The net current phasor is the sum of the branch current phasors $(\vec I = \vec I_R + \vec I_L)$\\
The peak current through the parallel combination is equal to $\mathcal{E}_{peak} / Z,$ where $Z$ is the impedance of the combination:
$$I = I_{peak} \cos (\omega t - \lvert \delta \rvert) \text{ where } I_{peak} = \frac{\mathcal{E}}{Z}$$
Based on the phasor diagram we have:
$$I_{peak}^2 = I_{R,peak}^2 + I_{L,peak}^2 = \left( \frac{\mathcal{E}_{peak}}{R} \right)^2 + \left( \frac{\mathcal{E}_{peak}}{X_L} \right)^2$$
$$= \mathcal{E}_{peak}^2\left( \frac{1}{R^2} + \frac{1}{X_L^2} \right) = \frac{\mathcal{E}_{peak}^2}{Z^2}$$
where $\frac{1}{Z^2} = \frac{1}{R^2} + \frac{1}{X_L^2}$. We do solve for $I_{peak}$ and yields:
$$I_{peak} = \frac{\mathcal{E}_{peak}}{Z} \text{ where } Z^{-2} = R^{-2} + X_L^{-2}$$
And finally, from the phasor diagram:
$$I = I_{peak} \cos (\omega t - \lvert \delta \rvert)$$
$$\text{where } \tan \lvert \delta \rvert = \frac{I_{L,peak}}{I_{R,peak}} = \frac{\frac{\mathcal{E}_{peak}}{X_L}}{\frac{\mathcal{E}_{peak}}{R}} = \fbox{$\frac{R}{X_L}$}$$

\paragraph{}
41 -  Figure 29-33 shows a load resistor that has a resistance of R L = 20.0 Omega connected to a high-pass filter consisting of an inductor that has inductance L = 3.20-mH and a resistor that has resistance R = 4.00-Omega. The output of the ideal ac generator is given by epsilon = (100 V) cos(2 pi ft). Find the rms currents in all three branches of the circuit if the driving frequency is (a) 500 Hz and (b) 2000 Hz. Find the fraction of the total average power supplied by the ac generator that is delivered to the load resistor if the frequency is (c) 500 Hz and (d) 2000 Hz.\\
\\
This equation reflects the voltage drops $\mathcal{E} = V_1 + V_2$, $V_1$ is the voltage drop across $R$ and $V_2$ is the voltage drop across the parallel combination of $L$ and $R_L$, if we take the same vectorial equation we have the relation for the phasors. The current for the parallel combination $\vec I = \vec I_{R_L} + \vec I_{L}.$ Also, $V_1$ is in phase with $I$ and $V_2$ is in phase with $I_{R_L}$. Draw the phasor diagram for the currents in the parallel combination, then add phasors for the voltages to the diagram.\\
We show the phasor diagram for the currents in the circuit and the modified diagram showing the voltage phasors.\\
The maximum current in the inductor $I_{2,peak}$ is given by:
$$I_{2,peak} = \frac{V_{2,peak}}{Z_2} \text{ where } Z_2^{-2} = R_L^{-2} + X_L^{-2}$$
the $\tan \lvert \delta \rvert$ is given by:
$$\tan \lvert \delta \rvert = \frac{I_{L,peak}}{I_{R,peak}} = \frac{V_{2,peak} / X_L}{V_{2,peak} / R_L}$$
$$= \frac{R_L}{X_L} = \frac{R_L}{\omega L} = \frac{R_L}{2\pi fL}$$
Solving for $\lvert \delta \rvert$:
$$\lvert \delta \rvert = \tan^{-1} \left( \frac{R_L}{2\pi fL} \right)$$
We apply the law of cosines to the triangle formed by the voltage phasors and obtain:
$$\mathcal{E}_{peak}^2 = V_{1,peak}^2 + V_{2,peak}^2 + 2V_{1,peak}V_{2,peak}\cos \lvert \delta \rvert$$
$$I_{peak}^2Z^2 = I_{peak}^2R^2 + I_{peak}^2Z_2^2 + 2I_{peak}RI_{peak}Z_2\cos \lvert \delta \rvert$$
Simplyfing for the current and Z:
$$Z^2 = R^2 + Z^2 + 2RZ_2\cos \lvert \delta \rvert$$
$$Z = \sqrt{R^2 + Z_2^2 + 2RZ_2\cos \lvert \delta \rvert}$$
The maximum current $I_{peak}$ in the circuit $I_{peak} = \frac{\mathcal{E}_{peak}}{Z}$\\
$I_{rms}$ is related to $I_{peak}$ according to: $I_{rms} = \frac{1}{\sqrt{2}}I_{peak}$\\
(a) We do substitute numerical values to get $\delta$:
$$\lvert \delta \rvert = \tan^{-1} \left( \frac{20.0\Omega}{2\pi (500Hz)(3.20mH)} \right) = \tan^{-1}\left( \frac{20.0\Omega}{10.053\Omega} \right) = 63.31^{\circ}$$
Solving for $Z_2$ gives:
$$Z_2^{-2} = R_L^{-2} + X_L^{-2} = \frac{1}{\sqrt{(20.0\Omega)^{-2}+(10.053\Omega)^{-2}}} = 8.982\Omega$$
For $Z$ we do substitute numerical values and evaluate:
$$Z = \sqrt{(4.00\Omega)^2 + (8.982\Omega)^2 + 2(4.00\Omega)(8.982\Omega)\cos 63.31^{\circ}} = 11.36\Omega$$
We do substitute values and evaluate for $I_{peak}$:
$$I_{peak} = \frac{100V}{11.36\Omega} = 8.806A$$
Once we have I peak, we do substitute for I rms:
$$I_{rms} = \frac{1}{\sqrt{2}}(8.806A) = \fbox{6.23A}$$
The maximum and rms values of $V_2$ are given by:
$$V_{2,peak} = I_{peak}Z_2 = (8.806A)(8.982\Omega) = 79.095V$$
$$V_{2,rms} = \frac{1}{\sqrt{2}}V_{2,peak} = \frac{1}{\sqrt{2}}(79.095V) = 55.929V,$$
The rms values of $I_{R_L,rms} \text{ and } I_{L,rms}$:
$$I_{R_L,rms} = \frac{V_{2,rms}}{R_L} = \frac{55.929V}{20.0\Omega} = \fbox{2.80A}$$
$$I_{L,rms} = \frac{V_{2,rms}}{X_L} = \frac{55.929V}{10.053\Omega} = \fbox{5.53A}$$
(b) We do proceed as in (a) with $f = 2000Hz$ to obtain:
$$X_L = 40.2\Omega, \lvert \delta \rvert = 26.4^{\circ}, Z_2 = 17.9\Omega , Z = 21.6\Omega , I_{peak} = 4.64A, \text{ and } I_{rms} = \fbox{3.28A}$$
$$V_{2,max} = 83.0V, V_{2,rms} = 58.7V, I_{R_L,rms} = \fbox{2.94A}, \text{ and } I_{L,rms} = \fbox{1.46A}$$
(c) The sum of the power dissipated in the two resistors equals the power delivered by the ac source. The fraction of the total power delivered by the source that is dissipated in load resistor is given by:
$$\frac{P_{R_L}}{P_{R_L} + P_R} = \left( 1 + \frac{P_R}{P_{R_L}} \right)^{-1} = \left( 1 + \frac{I_{rms}^2 R}{I_{R_L,rms}^2 R_L}\right)^{-1}$$
We substitute numerical values for $f = 500 Hz$:
$$\frac{P_{R_L}}{P_{R_L} + P_R}\bigg|_{f=500Hz} = \left( 1 + \frac{(6.23A)^2(4.00\Omega)}{(2.80A)^2(20.0\Omega)} \right)^{-1} = 0.502 = 50.2 \%$$
(d) Substitute numerical values for $f = 2000 Hz$ to obtain:
$$\frac{P_{R_L}}{P_{R_L} + P_R}\bigg|_{f=2000Hz} = \left( 1 + \frac{(3.28A)^2(4.00\Omega)}{(2.94A)^2(20.0\Omega)} \right)^{-1} = 0.800 = 80.0\%$$

\paragraph{}
42 - An ideal ac voltage source whose emf E 1 is given by (20 V) cos(2 pi ft) and an ideal battery whose emf E 2 is 16 V are connected to a combination of two resistors and an inductor (Figure 29-34), where R 1 = 10 Omega, R 2 = 8.0 Omega, and L = 6.0 mH. Find the average power delivered to each resistor if (a) the driving frequency is 100 Hz, (b) the driving frequency is 200 Hz, and (c) the driving frequency is 800 Hz.\\
\\
Let's treat the ac and dc components separately. L acts like a short circuit for the component. Denote the peak value of the voltage supplied by $\mathcal{E}_{1,peak}$, then use $P = \mathcal{E}_2^2 / R$ to find the power dissipated in the resistors by the current from the ideal battery. We apply Kirchhoff's loop including to $L, R_1, R_2$ to derive an expression for the average power delivered to each resistor by the ac voltage source.\\
(a) The total power delivered to $R_1$ and $R_2$:
\begin{equation} \label{p1}
  P_1 = P_{1,dc} + P_{1,ac}
\end{equation}
\begin{equation} \label{p2}
  P_2 = P_{2,dc} + P_{2,ac}
\end{equation}
The dc power delivered to the resistors whose resistances are $R_1$ and $R_2$:
$$P_{1,dc} = \frac{\mathcal{E}_2^2}{R_1} \text{ and } P_{2,dc} = \frac{\mathcal{E}_2^2}{R_2}$$
We express the average ac power delivered to $R_1$:
$$P_{1,ac} = \frac{\mathcal{E}_{1,rms}^2}{R_1} = \frac{\mathcal{E}_{1,peak}^2}{2R_1}$$
Applying Kirchhoff's loop rule clockwise to the loop includes $R_1, L, R_2$:
$$R_1I_1 - Z_2I_2 = 0$$
We solve for $I_2$:
$$I_2 = \frac{R_1}{Z_2}I_1 = \frac{R_1 \mathcal{E}_{1,peak}}{Z_2R_1} = \frac{\mathcal{E}_{1,peak}}{Z_2}$$
We express teh average ac power delivered to $R_2$:
$$P_{2,ac} = \frac{1}{2}I_{2,rms}^2R_2 = \frac{1}{2}\left( \frac{\mathcal{E}_{1,peak}}{Z_2} \right)^2 R_2 = \frac{\mathcal{E}_{1,peak}^2 R_2}{2Z_2^2}$$
Substituting in equations \ref{p1} and \ref{p2}: 
$$P_1 = \frac{\mathcal{E}_2^2}{R_1} + \frac{\mathcal{E}_{1,peak}^2}{2R_1}$$
$$P_2 = \frac{\mathcal{E}_2^2}{R_2} + \frac{\mathcal{E}_{1,peak}^2 R_2}{2Z_2^2}$$
Substituting numerical values and evaluate P:
$$P_1 = \frac{(16V)^2}{10\Omega} + \frac{(20V)^2}{2(10\Omega)} = \fbox{46W}$$
$$P_2 = \frac{(16V)^2}{8\Omega} + \frac{(20V)^2(8.0\Omega)}{2[(8\Omega)^2 + (2\pi \{100s^{-1}\}\{6.0mH\})^2]} = \fbox{52W}$$
(b) We proceed as in (a) to evaluate $P_1$ and $P_2$ with $f = 200 Hz$:
$$P_1 = 25.6W + 20.0W = \fbox{46W}$$
$$P_2 = 32.0W + 13.2W = \fbox{45W}$$
(c) We proceed as in (a) to evaluate $P_1$ and $P_2$ with $f = 800 Hz$:
$$P_1 = 25.6W + 20.0W = \fbox{46W}$$
$$P_2 = 32.0W + 1.64W = \fbox{34W}$$

\paragraph{}
43 - An ac circuit contains a resistor and an ideal inductor connected in series. The voltage rms drop across this series combination is 100-V and the rms voltage drop across the inductor alone is 80 V. What is the rms voltage drop across the resistor?\\
\\
The voltage across the resistor can be found using a phasor diagram. The phasor diagram is shown. Using the Pythagorean theorem we can express $V_R$:
$$V_R = \sqrt{\mathcal{E}_{rms}^2} - V_L^2 = \sqrt{(100V)^2 - (80V)^2} = \fbox{60V}$$

\section{Filters and Rectifiers}
\paragraph{}
44 - The circuit shown in Figure 29-35 is called an RC high-pass filter
because it transmits input voltage signals that have high frequencies with greater
amplitude than it transmits input voltage signals that have low frequencies. If the
input voltage is given by V in = V in peak cos omega t, show that the output voltage is
V out = V H cos( omega t – delta ) where V H = V in peak
1 + ( omega RC ) . (Assume that the output is - 2 connected to a load that draws only an insignificant amount of current.) Show that
this result justifies calling this circuit a high-pass filter.\\
\\
The phasor diagram for the RC high-pass filter is shown. $\vec V_{app}$ and $\vec V_R$ are the phasors for $V_{in}$ and $V_{out}$ respectively. Note that $\tan \delta = -X_C / R.$ That $\delta$ is negative follows from the fact that $\vec V_{app}$ lags $\vec V_R$ by $\lvert \delta \rvert$. The projection of $\vec V_{app}$ onto the horizontal axis is $V_{app} = V_{in},$ and the projection of $\vec V_R$ onto the horizontal axis is $V_R = V_{out}$. We start by expressing $V_{app}$:
$$V_{app} = V_{app,peak} \cos \omega t \text{ where } V_{app,peak} = I_{peak}Z$$
\begin{equation}\label{rsq}
\text{and } Z^2 = R^2 + X_C^2
\end{equation}
And as $\delta < 0$:
$$\omega t + \lvert \delta \rvert = \omega t - \delta$$
$V_R$ is given by:
$$V_R = V_{R,peak} \cos (\omega t - \delta) \text{ where } V_{R,peak} = V_H = I_{peak}R$$
Then we solve the equation \ref{rsq} for Z and substitute for $X_C$:
\begin{equation}\label{zsq}
  Z = \sqrt{R^2 + \left( \frac{1}{\omega C} \right)^2}
\end{equation}
Now, because $V_{out} = V_R$ we can express:
$$V_{out} = V_{R,peak} = \cos (\omega t - delta) = I_{in,peak}R\cos(\omega t - \delta) = \frac{V_{in peak}}{Z}R \cos (\omega t - \delta)$$
We now use equation \ref{zsq} to substitute for Z:
$$V_{out} = \frac{V_{in peak}}{\sqrt{R^2 + \left( \frac{1}{\omega C} \right)^2}}R \cos (\omega t - \delta)$$ 
Simplifying this expression:
$$V_{out} = \frac{V_{in peak}}{\sqrt{1 + (\omega RC)^{-2}}} \cos (\omega t - \delta) = \fbox{$V_H \cos (\omega t - \delta)$}$$
$$\text{where } V_H = \frac{V_{in peak}}{\sqrt{1+(\omega RC)^{-2}}}$$
As $\omega \rightarrow \infty$:
$$V_H \rightarrow \frac{V_{in peak}}{\sqrt{1+(0)^2}} = V_{in peak}$$
showing that the result is consistent with the highpass name for this circuit.

\paragraph{}
45 (a) Find an expression for the phase constant delta in Problem 44 in terms of omega , R and C. (b) What is the value of delta in the limit that omega tends 0? (c) What is the value of delta in the limit that omega tends infinity? (d) Explain your answers to Parts (b) and (c).\\
\\
The phasor diagram for the RC high-pass filter is shown. $\vec V_{app}$ and $V_R$ are the phasors for $V_{in}$ and $V_{out}$, respectively. The projection of $\vec V_{app}$ onto the horizontal axis is $V_{app} = V_{in}$, and the projection of $\vec V_R$ onto the horizontal axis is $V_R = V_{out}$.\\
(a) $\vec V_{app}$ lags $V_R$ by $\delta$:
$$\tan \delta = -\frac{V_C}{V_R} = -\frac{IX_C}{IR} = -\frac{X_C}{R}$$
Using the definition of $X_C$ we obtain:
$$\tan \delta = -\frac{\frac{1}{\omega C}}{R} = -\frac{1}{\omega RC}$$
and solving for $\delta$:
$$\delta = \fbox{$\tan^{-1} \left[ -\frac{1}{\omega RC} \right]$}$$
(b) As $\omega \rightarrow 0$:
$$\delta \rightarrow \fbox{$-90^{\circ}$}$$
(c) As $\omega \rightarrow \infty$:
$$\delta \rightarrow \fbox{0}$$

\paragraph{}
46 - Assume that in Problem 44, R = 20 kOmega and C = 15 nF. (a) At what frequency is V H = 1/2 V in peak ? This particular frequency is known as the 3 dB frequency, or f 3dB for the circuit. (b) Using a spreadsheet program, make a graph of log 10 (V H ) versus log 10 (f), where f is the frequency. Make sure that the scale extends from at least $10\%$ of the 3-dB frequency to ten times the 3-dB frequency. (c) Make a graph of delta versus log 10 (f) for the same range of frequencies as in Part (b). What is the value of the phase constant when the frequency is equal to the 3-dB frequency?\\
\\
The results found in problems 44 and 45 can be used to find $f_{3dB}$ and to plot graphs of $log(V_{out})$ versus $log(f)$ and $\delta$ versus $log(f)$.\\
(a) Using the result of problem 44 to express the ratio $V_{out} / V_{in peak}$:
$$\frac{V_{out}}{V_{in peak}} = \frac{\frac{V_{in peak}}{\sqrt{1 + (\omega RC)^{-2}}}}{V_{in peak}} = \frac{1}{\sqrt{1 + (\omega RC)^{-2}}}$$
And now when $V_{out} = V_{in peak} / \sqrt{2}$:
$$\frac{1}{\frac{1}{\sqrt{1 + (\omega RC)^{-2}}}} = \frac{1}{\sqrt{2}}$$
Squaring both sides of the equation and solving for $\omega RC$ to obtain:
$$\omega RC = 1 \Rightarrow \omega = \frac{1}{RC} \Rightarrow f_{3dB} = \frac{1}{2\pi RC}$$
We now substitute numerical values and evaluate:
$$f_{3dB} = \frac{1}{2\pi (20k\Omega)(15nF)} = \fbox{0.53kHz}$$
(b) Using result from problem 44 we have:
$$V_{out} = \frac{V_{in peak}}{\sqrt{1 + (\omega RC)^{-2}}}$$
And in problem 45 we showed that:
$$\delta = \tan^{-1} \left[ -\frac{1}{\omega RC} \right]$$
We rewrite these expressins in terms of $f_{3dB}$:
$$V_{out} = \frac{V_{in peak}}{\sqrt{1 + \left( \frac{1}{2\pi fRC} \right)^2}} = \frac{V_{peak}}{\sqrt{1 + \left( \frac{f_{3dB}}{f}\right)^2}}$$
and
$$\delta = \tan^{-1} \left[ -\frac{1}{2\pi fRC} \right] = \tan^{-1} \left[ -\frac{f_{3dB}}{f} \right]$$
A spreadsheet program to generate the data for a graph of $V_{out}$ versus $f$ and $\delta$ versus $f$ is shown, including formulas used to calculate the quantities in columns:
\begin{tabular}{| l | c | c |}
  \hline
  Cel & Formula/Content & Algebraic Form \\
  \hline
  B1 & 2.00E+03 & R \\
  \hline
  B2 & 1.50E-08 & C \\
  \hline
  B3 & 1 & $V_{in peak}$\\
  \hline
  B4 & 531 & $f_{3dB}$\\
  \hline
  A8 & 53 & $0.1f_{3db}$\\
  \hline
  C8 & \$B\$3/SQRT(1+(1(\$B\$4/A8))\^{} 2) & $\frac{V_{in peak}}{\sqrt{1 + \left( \frac{f_{3dB}}{f} \right)^2}}$\\
  \hline
  D8 & LOG(C8) & $log(V_{out}$\\
  \hline
  E8 & ATAN(-\$B\$4/A8) & $\tan^{-1}\left[ -\frac{f_{3dB}}{f} \right]$\\
  \hline
  F8 & E8*180/PI() & $\delta$ in degress\\
  \hline
\end{tabular}

The graph of $log(V_{out}$ versus $log(f)$ is shown for $V_{in peak} = 1V$.\\
A graph of $\delta$ in degrees as a function of $log(f)$ is shown.\\
As shown by the spreadsheet program, we can see that when $f = f_{3dB}$, $\delta \approx \fbox{$-44.9^{\circ}$}$. This result agrees with its calculated value of $\fbox{$-45.0^{\circ}$}$

\paragraph{}
47 - A slowly varying voltage signal V(t) is applied to the input of the high-pass filter of Problem 44. Slowly varying means that during one time constant (equal to RC) there is no significant change in the voltage signal. Show that under these conditions the output voltage is proportional to the time derivative of V(t). This situation is known as a differentiation circuit.\\
\\
Using the Kirchhoff's loop rule we can obtain a differential equation relating the input, capacitor, and resistor voltages. Because the voltage drop across the ressistor is small compared to the voltage drop across the capacitor, we can express the voltage drop across the capacitor in terms of the input voltage.\\
First apply Kirchhoff's loop rule to the input side of the filter:
$$V(t) - V_C -IR = 0$$
where $V_C$ is the potential difference across the capacitor.\\
The we substitute for $V(t)$ and I to obtain:
$$V_{in peak}\cos \omega t - V_C - R\frac{dQ}{dt} = 0$$
And because $Q = CV_C$:
$$\frac{dQ}{dt} = \frac{d}{dt}[CV_C] = C\frac{dV_C}{dt}$$
Substituting we obtain:
$$V_{peak} \cos \omega t - V_C - RC\frac{dVC}{dt} = 0$$
the differential equation describing the potential difference across the capacitor.\\
There is no significant change in the voltage signal during one time constant, so we can express:
$$\frac{dV_C}{dt} = 0 \Rightarrow RC\frac{dV_C}{dt} = 0$$
Substituting for $RC\frac{dV_C}{dt}$ yields:
$$V_{in peak} \cos \omega t - V_C = 0$$
and
$$V_C = V_{in peak}\cos \omega t$$
As a consequence, the potential difference across the resistor is given by:
$$V_R = RC\frac{dV_C}{dt} = \fbox{$RC\frac{d}{dt}[V_{in peak}\cos \omega t]$}$$

\paragraph{}
48 - We can describe the output from the high-pass filter from Problem 44 using a decibel scale: beta = ( 20 dB ) log 10 ( V H V in peak ) , where beta is the output in decibels. Show that for V H = 1 / sqrt2 V in peak , beta = 3.0 dB. The frequency at which VH = 1 / sqrt2 V in peak is known as f 3dB (the 3-dB frequency). Show that for $f << f 3dB$ , the output beta drops by 6 dB if the frequency f is halved.\\
\\
We use the expression found for $V_H$ in problem 44 and the definition of $\beta$ given in the problem statement to show that every time the frequency is halved, the output drops by 6 dB.\\
This equation we have from problem 44:
$$V_H = \frac{V_{in peak}}{\sqrt{1 + (\omega RC)^{-2}}}$$
or
$$\frac{V_H}{V_{in peak}} = \frac{1}{\sqrt{1 + (\omega RC)^{-2}}}$$
We need to express this ratio in terms of $f$ and $f_{3dB}$, we simplify as well:
$$\frac{V_H}{V_{peak}} = \frac{1}{\sqrt{1 + \left( \frac{f_{3dB}}{f} \right)^2}} = \frac{f}{\sqrt{f_{3dB}^2\left( \frac{f^2}{f_{3dB}^2} \right)}}$$
For $f << f_{3dB}$:
$$\frac{V_H}{V_{peak}} \approx \frac{f}{\sqrt{f_{3dB}^2 \left( 1 + \frac{f^2}{f_{3dB}^2} \right)}} = \frac{f}{f_{3dB}}$$
Using the $\beta$ definition we have:
$$\beta = 20log_{10} \left( \frac{V_H}{V_{peak}} \right)$$
Substituting for $V_H / V_{peak}$ we obtain:
$$\beta = 20log_{10} \left( \frac{f}{f_{3dB}} \right)$$
If we double the frequency we obtain:
$$\beta' = 20log_{10} \left( \frac{\frac{1}{2}f}{f_{3dB}} \right)$$
The decibel level change is given by:
$$\Delta \beta = \beta' - \beta = 20 log_{10}\left(\frac{\frac{1}{2}f}{f_{3dB}}\right) - 20log_{10}\left(\frac{f}{f_{3dB}}\right) = 20log_{10}(1/2) \approx \fbox{-6dB}$$

\paragraph{}
49 - Show that the average power dissipated in the resistor of the high-pass filter of Problem 44 is given by (formula)\\
\\
The instantaneous power dissipated in the resistor can be expressed, then using the fact that the average value of the square of the cosine function over one cycle is half to establish the given result.\\
The instantaneous power P(t) dissipated in the resistor is:
$$P(t) = \frac{V_{out}^2}{R}$$
The output voltage is:
$$V_{out} = V_H\cos (\omega t - \delta)$$
Taking from problem 44:
$$V_H = \frac{V_{in peak}}{\sqrt{1 + (\omega RC)^{-2}}}$$
Substitute in the expression for P(t):
$$P(t) = \frac{V_H^2}{R} \cos (\omega t - \delta) = \frac{V_{in peak}^2}{R[1 + (\omega RC)^{-2}]} \cos (\omega t - \delta)$$
And taking in consideration that the average value of the square of the cosine function over one cycle is one half:
$$P_{av} = \fbox{$\frac{V_{in peak}^2}{2R[1 + (\omega RC)^{-2}]}$}$$

\paragraph{}
50 - One application of the high-pass filter of Problem 44 is a noise filter
for electronic circuits (a filter that blocks out low-frequency noise). Using a resistance value of 20 k Omega, find a value for the capacitance for the high-pass filter that attenuates a 60-Hz input voltage signal by a factor of 10. That is, so V H = 1 / 10 V in peak.\\
\\
We should solve the expression for $V_H$ from problem 44 for the required capacitance of the capacitor.\\
Taking the expression from problem 44:
$$V_H = \frac{V_{in peak}}{\sqrt{1 + (\omega RC)^{-2}}}$$
And we do require that:
$$\frac{V_H}{V_{in peak}} = \frac{1}{\sqrt{1 + (\omega RC)^{-2}}} = \frac{1}{10}$$
or
$$\sqrt{1 + (\omega RC)^{-2}} = 10$$
If we do solve for C yields:
$$C = \frac{1}{\sqrt{99}\omega R} = \frac{1}{2\pi \sqrt{99}Rf}$$
Now we finally substitute numerical values and evaluate:
$$C = \frac{1}{2\pi(20k\Omega)(60Hz)} = \fbox{13nF}$$

\paragraph{}
51 - The circuit shown in Figure 29-36 is an example of a low-pass filter. (Assume that the output is connected to a load that draws only an insignificant amount of current.) (a) If the input voltage is given by V in = V in peak cos omega t, show that the output voltage is V out = V L cos( omega t – delta ) where V L = V in peak 1 + ( omega RC ) . (b) Discuss the trend of the output voltage in the limiting cases omega tends to 0 and infinity.\\
\\
The phasor diagram for the RC low-pass filter shows that $\vec V_{app}$ and $\vec V_C$ are the phasors for $V_{in}$ and $V_{out}$ respectively. The projection of $\vec V_{app}$ onto the horizontal axis is $V_{app} = V_{in}$, the projection of $\vec V_C$ onth the horizontal axis is $V_C = V_{out}$, $V_{peak} = \lvert \vec V_{app} \rvert$, and $\phi$ is the angle between $\vec V_C$ and the horizontal axis.\\
(a)
We express $V_{app}$:
$$V_{app} = V_{in peak} \cos \omega t \text{ where } V_{in peak} = I_{peak}Z$$
\begin{equation}\label{Zand}
  \text{ and } Z^2 = R^2 + X_C^2
\end{equation}
$V_{out} = V_C$ is given by:
$$V_{out} = V_{C,peak} = \cos \phi = I_{peak}X_C \cos \phi$$
Now we define $\delta$ as shown in the phasor diagram:
$$V_{out} = I_{peak}X_C \cos (\omega t - \delta) = \frac{V_{in peak}}{Z}X_C \cos (\omega t - \delta)$$
Solving equation \ref{Zand} for Z and substituting $X_C$:
\begin{equation}\label{zsqrt}
  Z = \sqrt{R^2 + \left( \frac{1}{\omega C} \right)^2}
\end{equation}
We can use equation \ref{zsqrt} to substitute for Z and substitute for $X_C$:
$$V_{out} = \frac{V_{in peak}}{\sqrt{R^2 + \left( \frac{1}{\omega C} \right)^2}} \left( \frac{1}{\omega C} \right) \cos (\omega t - \delta)$$
If we simplify further we can obtain:
$$V_{out} = \frac{V_{in peak}}{\sqrt{1 + (\omega RC)^2}} \cos (\omega t - \delta)$$
or
$$V_{out} = \fbox{$V_L \cos (\omega t - \delta)$} \text{ where } V_L = \fbox{$\frac{V_{in peak}}{\sqrt{1 + (\omega RC)^2}}$}$$
(b) We can note that as $\omega \rightarrow 0, V_L \rightarrow V_{peak}$. This is physically feasible as for low frequencies $X_C$ is large and hence a larger peak input voltage will appear across it than appears across it for high frequencies.\\
Yet another note is that as $\omega \rightarrow \infty, V_L \rightarrow 0$. This is physically feasible since, for high frequencies, $X_C$ is small and, hence, a smaller peak voltage will appear across it than appears it for low frequencies.\\
Remarks: In Figures 29-19 and 29-20, $\delta$ is defined as the phase of the
voltage drop across the combination relative to the voltage drop across
the resistor.

\paragraph{} 
52 (a) Find an expression for the phase angle delta for the low-pass filter of Problem 51 in terms of omega , R and C. (b) Find the value of delta in the limit that omega tends 0 and in the limit that omega tends infinity. Explain your answer.\\
\\
The phasor diagram for the RC low-pass filter is shown. $\vec V_{app}$ and $\vec V_C$ are the phasors for $V_{in}$ and $V_{out}$ respectively. The projection of $\vec V_{app}$ onto the horizontal axis is $V_{app} = V_{in}$ and the projection of $\vec V_C$ onto the horizontal axis is $V_C = V_{out}$. $V_{peak} = \lvert \vec V_{app} \rvert$.\\
(a) Analyzing the phasor diagram we have:
$$\tan \delta = \frac{V_R}{V_C} = \frac{I_{peak}R}{I_{peak}X_C} = \frac{R}{X_C}$$
and from the $X_C$ definition we obtain:
$$\tan \delta = \frac{R}{\frac{1}{\omega C}} = \omega RC$$
Solving for $\delta$:
$$\delta = \fbox{$\tan^{-1}(\omega RC)$}$$
(b) As $\omega \rightarrow 0, \delta \rightarrow \fbox{$0^{\circ}$}$. This behaviour makes sense physically in that, at low frequencies, $X_C$ is very large compared to R and, as a consequence, $V_C$ is in phase with $V_{in}$.\\
As $\omega \rightarrow \infty, \delta \rightarrow \fbox{$90^{\circ}$}$. This behaviour makes sense physically in that, at high frequencies, $X_C$ is very small compared to R and, as a consequence, $V_C$ is out of phase with $V_{in}$\\
Remarks: See the spreadsheet solution in the following problem for additional evidence that our answer for Part (b) is correct.

\paragraph{}
53 Using a spreadsheet program, make a graph of V L versus input frequency f and a graph of phase angle delta versus input frequency for the low-pass filter of Problems 51 and 52. Use a resistance value of 10 kOmega and a capacitance value of 5.0 nF.\\
\\
In problems 51 and 52 we derived expressions for $V_L$ and $\delta$ to plot graphs of $V_L$ versus $f$ and $\delta$ versus $f$ for the low-pass filter of problem 51. We express $V_L$ and $\delta$ as functions of $f_{3dB}$ to simplify the spreadsheet program.\\
From problems 51 and 52 we have:
$$V_L = \frac{V_{in peak}}{\sqrt{1 + (\omega  RC)^2}}$$
and
$$\delta = \tan^{-1} (\omega RC)$$
We express these expressions in terms of $f$ and obtain:
$$V_L = \frac{V_{in peak}}{\sqrt{1 + (2\pi fRC)^2}}$$
and
$$\delta = \tan^{-1} (2\pi fRC)$$
In the table we show a spreadsheet program to generate the data for graphs of $V_L$ versus $f$ and $\delta$ versus $f$ for the low-pass filter. $V_{in peak}$ has been arbitrarily set equal to 1. The formulas used to calculate the quantities are shown in the columns:\\
\begin{tabular}{| l | c | c |}
  \hline
  Cel & Formula/Content & Algebraic Form \\
  \hline
  B1 & 2.00E+03 & R \\
  \hline
  B2 & 1.50E-09 & C \\
  \hline
  B3 & 1 & $V_{in peak}$\\
  \hline
  B8 & \$B\$3/SQRT(1+((2*PI()*A8*1000*\$B\$1*\$B\$2)\^{}2)) & $\frac{V_{in peak}}{\sqrt{1 + (2\pi fRC)^2}}$\\
  \hline
  C8 & ATAN(2*PI()*A8*1000*\$B\$1*\$B\$2) & $\tan^{-1} (2\pi fRC)$\\
  \hline
  D8 & C8*180/PI() & $\delta$ in degrees\\
  \hline
\end{tabular}
A graph of $V_L$ as function of $f$ is shown:\\
A graph of $\delta$ as a function of $f$ is shown:\\

\paragraph{}
54 - A rapidly varying voltage signal V(t) is applied to the input of the low-pass filter of Problem 51. Rapidly varying means that during one time constant (equal to RC) there are significant changes in the voltage signal. Show that under these conditions the output voltage is proportional to the integral of V(t) with respect to time. This situation is known as an integration circuit.\\
\\
The Kirchhoff's loop rule can be used to find a differential equation relating the input, capacitor, and resistor voltages. Then we assume a solution for this equation that is a linear combination of sine and cosine terms with coefficients that we can find by substitution in the differential equation. The solution of these simultaneous equations will yield the amplitude of the output voltage.\\
First we apply Kirchhoff's loop rule to the input side of the filter to obtain $V(t) - IR V_C = 0$, where $V_C$ is the potential difference across the capacitor.\\
We substitute for $V(t)$ and I:
$$V_{in peak} \cos \omega t - R\frac{dQ}{dt} - V_C = 0$$
And because $Q = CV_C$:
$$\frac{dQ}{dt} = \frac{d}{dt}[CV_C] = C\frac{dV_C}{dt}$$
Sustituting for $dQ / dt$:
$$V_{in peak} \cos \omega t - RC \frac{dV_C}{dt} - V_C = 0$$
the differential equation describing the potential difference across the capacitor.\\
$V_C$ is given by $V_C = IX_C = \frac{1}{\omega C}$. $V(t)$ varies rapidly meaning that $\omega >> 1$, therefore, $V_C \approx 0$, and:
$$V_{peak} \cos \omega t -RC \frac{dV_C}{dt} = 0$$
Separating variables in the differential equation and solve for $V_C$:
$$V_C = \fbox{$\frac{1}{RC}\int V_{peak} \cos \omega t dt$}$$

\paragraph{}
55 - The circuit shown in Figure 29-37 is a trap filter. (Assume that the output is connected to a load that draws only an insignificant amount of current.) (a) Show that the trap filter acts to reject signals in a band of frequencies centered at omega = 1/ LC . (b) How does the width of the frequency band rejected depend on the resistance R?\\
\\
The phasor diagram for the trap filter is shown. $\vec V_{app}$ and $\vec V_L + \vec V_C$ are the phasors for $V_{in}$ and $V_{out}$ respectively. The projection of $\vec V_{app}$ onto the horizontal axis is $V_{app} = V_{in}$, and the projection of $\vec V_L + \vec V_C$ onto the horizontal axis is $V_L + V_C = V_{out}$. We assume the impedance of the trap to be zero, then the frequency at which the circuits rejects signals will be shown. If we define $\Delta \omega = \lvert \omega - \omega_{trap} \rvert$ and do we require that $\lvert Z_{trap} \rvert = R$ will yield and expression for the bandwidth and reveal its dependence on R.\\
Expressing $V_{app}$:
$$V_{app} = V_{app, peak} \cos \omega t$$
where $V_{app, peak} = V_{peak} = I_{peak}Z$ and
\begin{equation}\label{zpeaks}
  Z^2 = R^2 + (X_L - X_C)^2
\end{equation}
On the other hand, $V_{out}$ is given by:
$$V_{out} = V_{out, peak} \cos (\omega t - \delta)$$
where $V_{out, peak} = I_{peak}Z_{trap}$ and $Z_{trap} = X_L - X_C$.\\
Solving for Z in equation \ref{zpeaks} yields:
\begin{equation}\label{zsqrt}
  Z^2 = \sqrt{R^2 + (X_L - X_C)^2}
\end{equation}
And because $V_{out} = V_L + V_C$:
$$V_{out} = V_{out, peak} \cos (\omega t - \delta) = I_{peak}Z_{trap} \cos (\omega t - \delta)$$
$$= \frac{V_{peak}}{Z}Z_{trap} \cos (\omega t - \delta)$$
We can use equation \ref{zsqrt} to substitute for Z:
$$V_{out} = \frac{V_{peak}}{\sqrt{R^2 + Z_{trap}^2}} Z_{trap} \cos (\omega t - \delta)$$
Provided that $Z_{trap} = 0$, we note that $V_{out} = 0$, then set $Z_{trap} = 0$ and obtain $Z_{trap} = X_L - X_C = 0$. Substituting for $X_L$ and $X_C$ yields:
$$\omega L - \frac{1}{\omega C} = 0 \Rightarrow \omega = \fbox{$\frac{1}{\sqrt{LC}}$}$$
(b) We have as bandwith:
\begin{equation}\label{bwt}
  \Delta \omega = \lvert \omega - \omega_{trap} \rvert
\end{equation}
We use bandwith definition involving the frequency at which $\lvert Z_{trap} \rvert = R$. Then:
$$\omega L - \frac{1}{\omega C} = R \Rightarrow \omega^2 LC - 1 = \omega RC$$
And because $\omega_{trap} = 1 / \sqrt{LC}$:
$$\left( \frac{\omega}{\omega_{trap}} \right)^2 - 1 = \omega RC$$
We solve for $\omega^2 - \omega_{trap}^2$:
$$\omega^2 - \omega_{trap}^2 = (\omega - \omega_{trap})(\omega + \omega_{trap})$$
And because $\omega \approx \omega_{trap}, \omega + \omega_{trap} \approx 2\omega_{trap}$:
$$\omega^2 - \omega_{trap}^2 \approx 2\omega_{trap}(\omega - \omega_{trap})$$
Now substitute in equation \ref{bwt} to obtain:
$$\Delta \omega = \lvert \omega - \omega_{trap} \rvert = \frac{RC\omega_{trap}^2}{2} = \fbox{$\frac{R}{2L}$}$$

\paragraph{}
56 - The output voltage will mirror input voltage minus a 0.60 V drop for voltages greater than 0.60 V. When the voltage is below 0.60 V, the output voltage will be zero. A spreadsheet program was utilized to plot the following graph. The peak voltage and angular frequency were both arbitrarily set equal to one.\\

\paragraph{}
57 - The time constant for the RC circuit and the frequency of the input signal can be related through the use of the potential difference decay across the capacitor. An approximate value for C can be found by expanding the exponential factor in the expression for $V_C$. The C value will limit the variation in the output voltage by less than 50 percent.\\
We find the voltage across the capacitor:
$$V_C = V_{in} e^{-t / RC}$$
If we expend the exponential factor:
$$e^{-t / RC} \approx 1 - \frac{1}{RC}t$$
And for a decay of less than 50 percent:
$$1 - \frac{1}{RC}t \le 0.5 \Rightarrow C \le \frac{2}{R}t$$
And finally, the voltage is positive at every cycle, $t = 1 / 60 s$:
$$C \le \frac{2}{1.00k\Omega}\left( \frac{1}{60}s \right) = \fbox{$33\mu F$}$$

\paragraph{}
58 - The current lags the voltage across the inductor and leads the voltage across a capacitor. If we use $I_{L, peak} = \mathcal{E}_{peak} / X_L$ and $I_{C, peak} = \mathcal{E}_{peak}X_C$ to find the amplitudes for these currents. The current in the generator is zero under resonance conditions, that is, when $\lvert I_L \rvert = \lvert I_C \rvert$. For finding the currents in the inductor and capacitor at resonance, we can use the shared potential difference across them and their reactances together with our knowledge of the phase relationships just mentioned.\\
(a) We do express the currents amplitudes through the inductor and capacitor:
$$I_{L_ peak} = \frac{\mathcal{E}_{peak}}{X_L} = \frac{\mathcal{E}_{peak}}{2\pi fL}$$
and
$$I_{C_ peak} = \frac{\mathcal{E}_{peak}}{X_C} = \frac{\mathcal{E}_{peak}}{\frac{1}{2\pi fC}} = 2 \pi fC\mathcal{E}_{peak}$$
Substituting numerical values:
$$I_{L, peak} = \frac{100V}{(4.00H)2 \pi f} = \fbox{$\frac{25.0V/H}{2 \pi f}, \text{ lagging } \mathcal{E} \text{ by } 90^{\circ}$}$$
and
$$I_{C, peak} = (25.0 \mu F)(100V) \omega = \fbox{$(2.5mV \cdot F)2 \pi f, \text{ leading } \mathcal{E} \text{ by } 90^{\circ}$}$$
(b) We do express the condition that $I = 0$:
$$\lvert I_L \rvert = \lvert I_C \rvert \text{ or } \frac{\mathcal{E}}{\omega L} = \frac{\mathcal{E}}{\frac{1}{\omega C}} = \omega C\mathcal{E} \Rightarrow \omega = \frac{1}{\sqrt{LC}}$$
Now we substitute numerical values and evaluate $\omega$:
$$\omega = \frac{1}{\sqrt{(4.00H)(25.0 \mu F)}} = \fbox{100 rad/s}$$
We use numerical values expressing the current in the inductor at $\omega = \omega_0$:
$$I_L = \left( \frac{25.0V/H}{100s^{-1}} \right) \cos \left( \omega t - \frac{\pi}{2} \right) = \fbox{$(250mA) \cos \left( \omega t - \frac{\pi}{2} \right)$}$$
where $\omega = 100 rad / s$
We also express the current in the capacitor at $\omega = \omega_0$:
$$I_C = (2.5mV \cdot F)(100s^{-1}) \cos \left( \omega t + \frac{\pi}{2} \right)$$
$$= \fbox{$-(250mA) \cos \left( \omega t + \frac{\pi}{2} \right)$}$$
where $\omega = 100 rad / s$.\\
(d) We show the phasor diagram for the case where the inductive reactance is larger than the capacitive reactance.

\paragraph{}
59 - If we differentiate with respect to time we can find I as a function of time. For (b), C can be found using $\omega = 1 / \sqrt{LC}$. The energy stored in the magnetic field of the inductor is given by $U_m = \frac{1}{2}LI^2$ and the energy stored in the electric field of the capacitor is given by $U_e = \frac{1}{2}\frac{Q^2}{C}$.\\
(a) We differentiate the charge with respect to time for obtaining the current:
$$I(t) = \frac{dQ}{dt} = \frac{d}{dt}\left[ (15 \mu C) \cos \left( \omega t + \frac{\pi}{4} \right) \right] = -(15 \mu C)(1250s^{-1}) \sin \left( \omega t + \frac{\pi}{4} \right)$$
$$-(18.75mA) \sin \left( \omega t + \frac{\pi}{4} \right) = \fbox{$-(19mA) \sin \left( \omega t + \frac{\pi}{4} \right)$}$$
where $\omega = 1250 rad / s$\\
(b) We have a relation for C, L and $\omega$, solving for C and evaluating at numerical values:
$$\omega = \frac{1}{\sqrt{LC}} \Rightarrow C = \frac{1}{\omega^2 L} = \frac{1}{(1250s^{-1})(28mH)} = 22.86 \mu F = \fbox{$23 \mu F$}$$
(c) For the magnetic energy at time t we have:
$$U_m(t) = \frac{1}{2}LI^2 = \frac{1}{2}(28mH)(18.75mA)^2 \sin^2 \left( \omega t + \frac{\pi}{4} \right) = \fbox{$(4.9 \mu J) \sin^2 \left( \omega t + \frac{\pi}{4} \right)$}$$
where $\omega = 1250 rad / s$.\\
We now find the electrical energy stored in the capacitor $U_e = \frac{1}{2} \frac{Q^2}{C}$ as a function of time:
$$U_e(t) = \frac{1}{2} \frac{(15 \mu F)^2}{22.86 \mu F} \cos^2 \left( \omega t + \frac{\pi}{4} \right)$$
$$= (4.92 \mu J) \cos^2 \left( \omega t + \frac{\pi}{4} \right)$$
where $\omega = 1250 rad / s$.\\
The total energy sum stored in electric and magnetic field is the sum of $U_e(t)$ and $U_m(t)$:
$$U = (4.92 \mu J) \sin^2 \left( \omega t + \frac{\pi}{4} \right) + (4.92 \mu J) \cos^2 \left( \omega t + \frac{\pi}{4} \right) = \fbox{$4.9 \mu J$}$$

\paragraph{}
60 - The capacitance of a dielectric field capacitor definition and the expression for the resonance frequency of an LC circuit can be used to derive an expression for the fractional change in the thicknes of the dielectric in terms of the resonance frequency and the frequency of the circuit when the dielectric is under copression. Afterwards, we can use this expression for $\Delta t / t$ to calculate the Young's modulus for the dielectric material.\\
We begin by using th definition for the Young's modulus of the dielectric material:
\begin{equation}\label{my}
  Y = \frac{stress}{strain} = \frac{\Delta P}{\Delta t / t}
\end{equation}
We let t to be the initial thickness of the dielectric, expressing the initial capacitance of the capacitor:
$$C_0 = \frac{\kappa \epsilon_0 A}{t}$$
We express the capacitance of the capacitor when it is under compression:
$$C_C = \frac{\kappa \epsilon_0 A}{t - \Delta t}$$
Now we express the resonance frequency of the capacitor before the dielectric is compressed:
$$\omega_0 = \frac{1}{\sqrt{C_0L}} = \frac{1}{\sqrt{\frac{\kappa \epsilon_0 AL}{t}}}$$
And now when the dielectric is compressed:
$$\omega_C = \frac{1}{\sqrt{C_CL}} = \frac{1}{\sqrt{\frac{\kappa \epsilon_0 AL}{t - \Delta t}}}$$
Simplifying by expressing the ratio of $\omega_C$ and $\omega_0$:
$$\frac{\omega_C}{\omega_0} = \frac{\sqrt{\frac{\kappa \epsilon_0 AL}{t}}}{\sqrt{\frac{\kappa \epsilon_0 AL}{t - \Delta t}}} = \sqrt{1 - \frac{\Delta t}{t}}$$
We can expand the radical binomially to obtain:
$$\frac{\omega_C}{\omega_0} = \left( 1 - \frac{\Delta t}{t} \right)^{1/2} \approx 1 - \frac{\Delta t}{2t}$$
provided that $\Delta t << t$.\\
Solving for $\Delta t / t$:
$$\frac{\Delta t}{t} = 2 \left( 1 - \frac{\omega_c}{\omega_0} \right)$$
Subsituting in equation \ref{my}:
$$Y = \frac{\Delta P}{2 \left( 1 - \frac{\omega_c}{\omega_0} \right)}$$
And finally using the numerical values we evaluate Y:
$$Y = \frac{(800atm)(102.325kPa/atm)}{2 \left( 1 - \frac{116MHz}{120MHz} \right)} = \fbox{$1.22 \times 10^9 N / m^2$}$$

\paragraph{}
61 - The capacitor can be modeled as the equivalent of two capacitors connected in parallel. Let $C_1$ by the capcacitance of the dielectric-filled capacitor and $C_2$ be the air-filled capacitor. We will derive expressions for each capacitance and then add them together to obtain $C(x)$. We can then use the given resonance frequency when $x = w / 2$ and the given value for L to evaluate $C_0$. In part (b) we can use our result for $C(x)$ and the relationship between $f, L$ and $C(x)$ at resonance to express $f(x)$.\\
(a) Let's express the equivalent capacitance of the two capacitors in parallel:
\begin{equation}\label{cx}
  C(x) = C_1 + C_2 = \frac{\kappa \epsilon_0 A_1}{d} + \frac{\epsilon_0 A_2}{d}
\end{equation}
We need to express $A_2$ in terms of the total area of a capacitor plate A, w, and the distance x:
$$\frac{A_2}{A} = \frac{x}{w} \Rightarrow A_2 = A \frac{x}{w}$$
We do express $A_1$ in terms of A and $A_2$:
$$A_1 = A - A_2 = A \left( 1 - \frac{x}{w} \right)$$
Substituting in equation \ref{cx} and simplifying:
$$C(x) = \frac{\kappa \epsilon_0 A}{d}\left( 1 - \frac{x}{w} \right) + \frac{\epsilon_0 A x}{d w}$$
$$= \frac{\epsilon_0 A}{d} \left[ \kappa \left( 1 - \frac{x}{w} \right) + \frac{x}{w} \right] = \kappa C_0 \left[ 1 - \frac{\kappa - 1}{kw} x \right]$$
where $C_0 = \frac{\epsilon_0 A}{d}$
We find $C(w / 2)$:
$$C \left( \frac{w}{2} \right) = \kappa C_0 \left[ 1- \frac{(\kappa - 1)w}{2kw} \right] = \kappa C_0 \left[ 1 - \frac{\kappa - 1}{2\kappa} \right] = C_0 \frac{\kappa + 1}{2}$$
Now for the resonance frequency of the circuit in terms of L and $C(x)$ we express:
\begin{equation}\label{fx}
  f(x) = \frac{1}{2 \pi \sqrt{LC(x)}}
\end{equation}
We do evaluate $f(w / 2)$:
$$f \left( \frac{w}{2} \right) = \frac{1}{2 \pi \sqrt{LC_0 \frac{\kappa + 1}{2}}} = \frac{1}{2 \pi} \sqrt{\frac{2}{(\kappa + 1)LC_0}}$$
Solving for $C_0$ we obtain:
$$C_0 = \frac{1}{2 \pi^2 f^2 \left( \frac{w}{2} \right)L(\kappa + 1)}$$
Now we substitute numerical values and evaluate $C_0$:
$$C_0 = \frac{1}{2 \pi^2 (90MHz)^2 \left( \frac{20cm}{2} \right)(2.0mH)(4.8 + 1)} = \fbox{5.4fF}$$
(b) Substitute for $C(x)$ in equation \ref{fx}:
$$f(x) = \frac{1}{2 \pi \sqrt{L\kappa C_0 \left[ 1 - \frac{\kappa - 1}{\kappa w} x \right]}}$$
And finally substitute numerical values and evaluate $f(x)$:
$$f(x) = \frac{1}{2 \pi \sqrt{(2.0mH)(4.8)(5.39 \times 10^{-16}F) \left[ 1 - \frac{4.8 - 1}{4.8(0.20m)} x \right]}} = \fbox{$\frac{70MHz}{\sqrt{1-(4.0m^{-1})x}}$}$$
\section{Driven RLC Circuits}

\paragraph{}
62 - In the diagram is shown the relationship between $\delta, X_L, X_C, R$. This reference triangle can be used to express the power factor for the given circuit. In (b) we can find the rms current from the rms potential difference and the impedance of the circuit. The rms current and the resistance of the resistor can be used to find the average power delivered by the source.\\
(a) The power factor is defined as:
$$\cos \delta = \frac{R}{Z} = \frac{R}{\sqrt{R^2 + (X_L - X_C)^2}}$$
We have no inductance in the circuit, therefore $X_L = 0$ and
$$\cos \delta = \frac{R}{\sqrt{R^2 + X_C^2}} = \frac{R}{\sqrt{R^2 + \frac{1}{\omega^2 C^2}}}$$
Substituting numerical values to evaluate $\cos \delta$:
$$\cos \delta = \frac{80\Omega}{\sqrt{(80\Omega)^2 + \frac{1}{(400s^{-1})^2(20 \mu F)^2}}} = \fbox{0.54}$$
(b) For this exercise, express the rms current in the circuit:
$$I_{rms} = \frac{\mathcal{E}_{rms}}{Z} = \frac{\frac{\mathcal{E}_{max}}{\sqrt{2}}}{\sqrt{R^2 + X_C}}$$
$$\frac{\mathcal{E}_{max}}{\sqrt{2} \sqrt{R^2 + \frac{1}{\omega^2 C^2}}}$$
Substituting numerical values and evaluate $I_{rms}$:
$$I_{rms} = \frac{20V}{\sqrt{2}\sqrt{(80\Omega)^2 + \frac{1}{(400s^{-1})^2(20 \mu F)^2}}} = 95.3mA = \fbox{95mA}$$
(c) The average power that the generator delivers is $P_{av} = I_{rms}^2R$, substituting numerical values and evaluating $P_{av}$:
$$P_{av} = (95.3mA)^2(80\Omega) = \fbox{0.73W}$$

\paragraph{}
63 - $Z = \sqrt{R^2 + (X_L - X_C)^2}$ gives the immpedance of an ac circuit. If we make $X_L = X_C = 0$ and then $R = 0$, we can evaluate the impedance expression for $P_{av}$.\\
(a) For $X = 0, Z = R$:
$$P_{av} = \frac{R\mathcal{E}_{rms}^2}{Z^2} = \frac{R\mathcal{E}_{rms}^2}{R^2} = \fbox{$\frac{R\mathcal{E}_{rms}^2}{R}$}$$
(b) and (c). If $R = 0$, then:
$$P_{av} = \frac{R\mathcal{E}_{rms}^2}{Z^2} = \frac{(0)\mathcal{E}_{rms}^2}{(X_L - X_C)^2} = \fbox{0}$$
Remarks: in an ideal inductor or capacitor there is no energy dissipation.

\paragraph{}
64 - The resonant frequency of the circuit can be found using $\omega_0 = 1 / \sqrt{LC}$, the rms current at resonance can be found using $I_{rms} = \mathcal{E}_{rms} / R$. We can find the reactances at $\omega = 8000 rad / s$, by using the definitions of $X_C, X_L$. The definitions of  $Z, I_{rms}$ to find the impedance and rms current at $\omega = 8000 rad / s$, and the definition of the phase angle to find $\delta$.\\
(a) We begin by expressing the resonant frequency $\omega_0$ in terms of L and C and substitute numerical values, evaluating:
$$\omega_0 = \frac{1}{\sqrt{LC}} = \frac{1}{\sqrt{(10mH)(2.0 \mu F)}} = \fbox{$7.1 \times 10^3 rad / s$}$$
(b) We need to find a relationship between the rms current at resonance and $\mathcal{E}_{rms}$ and the impedance of the circuit at resonance as well:
$$I_{rms} = \frac{\mathcal{E}_{rms}}{R} = \frac{\mathcal{E}_{max}}{\sqrt{2}R} = \frac{100V}{\sqrt{2}()5.0\Omega} = \fbox{14A}$$
(c) We do express and evaluate $X_C$ and $X_L$ at $\omega = 8000 rad / s$:
$$X_C = \frac{1}{\omega C} = \frac{1}{(8000s^{-1})(2.0 \mu F)} = 62.50 \Omega = \fbox{$63 \Omega$}$$
and
$$X_L = \omega L = (8000 s^{-1})(10 mH) = \fbox{$80\Omega$}$$
(d) The impedance needs to be expressed in terms of reactances, and substitute results from (c) and evaluate Z:
$$Z^2 = \sqrt{R^2 + (X_L - X_C)^2} = \sqrt{(5.0 \Omega)^2 + (80 \Omega - 62.5 \Omega)^2} = 18.2 \Omega = \fbox{$18 \Omega$}$$
(e) We need to relate the rms current at $\omega = 8000 rad / s$ to $\mathcal{E}_{rms}$ and the impedance of the circuit at this same frequency:
$$I_{rms} = \frac{\mathcal{E}_{rms}}{Z} = \frac{\mathcal{E}_{max}}{\sqrt{2}Z} = \frac{100V}{\sqrt{2}(18.2 \Omega)} = \fbox{3.9 A}$$
(f) We find $\delta$ as follows:
$$\delta = \tan^{-1} \left( \frac{X_L - X_C}{R} \right) = \tan^{-1} \left( \frac{80 \Omega - 62.5 \Omega}{5.0 \Omega} \right) = \fbox{$74^{\circ}$}$$

\paragraph{}
65 - $Q = \omega L / R$ gives us the Q factor of the circuit, the resonance width by $\Delta f = f_0 / Q = \omega_0 / 2 \pi Q$, and the power factor by $\cos \delta = R / Z$. Z is frequency dependent, hence we need to find $X_C$ and $X_L$ at $\omega = 8000 rad / s$ to be able to evaluate $\cos \delta$.\\
Let's use above definitions to express Q factor and resonance width of the circuit:
\begin{equation}\label{cu}
  Q = \frac{\omega_0 L}{R}
\end{equation}
and
\begin{equation}\label{delf}
 \Delta f = \frac{f_0}{Q} = \frac{\omega_0}{2 \pi Q}
\end{equation}
(a) We begin by expressin the resonance frequency for the circuit:
$$\omega_0 = \frac{1}{\sqrt{LC}}$$
Substituting in equation \ref{cu}:
$$Q = \frac{1}{\sqrt{LC} R} = \frac{1}{r}\sqrt{\frac{L}{C}}$$
Substituting numerical values and evaluating:
$$Q = \frac{1}{5.0 \Omega} \sqrt{\frac{10 mH}{2.0 \mu F}} = 14.1$$
Substitute numerical values in equation \ref{delf} and evaluate:
$$\Delta f = \frac{7.07 \times 10^3 rad / s}{2 \pi (14.1)} = \fbox{80 Hz}$$
(c) The power factor of the circuit is given by:
$$\cos \delta = \frac{R}{Z} = \frac{R}{\sqrt{R^2 + (X_L - X_C)^2}} = \frac{R}{\sqrt{R^2 + \left( \omega L - \frac{1}{\omega C}\right)^2}}$$
Substituting numerical values an evaluating for $\cos \delta$:
$$\cos \delta = \frac{5.0 \Omega}{\sqrt{(5.0 \Omega)^2 + \left( (8000s^{-1})(10 mH) - \frac{1}{(8000s^{-1})(2.0 \mu F)} \right)^2}} = \fbox{0.27}$$

\paragraph{}
66 - The Q factor for the circuit can be found from its definition $Q = f_0 / \Delta f$. Only substitute the numerical values.
$$Q = \frac{100.1 MHz}{0.050 MHz} \approx \fbox{$2.0 \times 10^3$}$$

\paragraph{}
67 - From the current definition $I_{peak} = \mathcal{E}_{peak} / Z$ to find the current in the coil and the definition of the phase angle to evaluate $\delta$. We can equate $X_L$ and $X_C$ to find the capacitance required so that the current and the voltage are in phase. $V_C =IX_C$ can be used to find the measured voltage across the capacitor.\\
(a) We express the current in the coil in terms of the potential difference across it and its impedance:
$$I_{peak} = \frac{\mathcal{E}_{peak}}{Z} = \frac{100 V}{10 \Omega} = \fbox{10 A}$$
(b) The phase angle $\delta$ is given by the next definition, and substitute numerical values to evaluate:
$$\delta = \cos^{-1} \left( \frac{R}{Z} \right) = \sin^{-1} \left( \frac{X_L}{Z} \right) = \sin^{-1} \left( \frac{8.0 \Omega}{10 \Omega} \right) = \fbox{$53^{\circ}$}$$
(c) Expressing the condition on the on the reactances that must be satisfied if the current and voltage are to be in phase:
$$X_L = X_C = \frac{1}{\omega C} \Rightarrow C = \frac{1}{\omega X_L} = \frac{1}{2 \pi fX_L}$$
Substituting numerical values:
$$C = \frac{1}{2 \pi (60s^{-1})(8.0 \Omega)} = 332 \mu F = \fbox{0.33 mF}$$
(d) Expressing the potential difference across the capacitor:
$$V_C = I_{peak}X_C$$
And relate the peak current in the circuit to the impedance of the circuit when $X_L = X_C$:
$$I_{peak} = \frac{V_{peak}}{R}$$
Substituting for the current:
$$V_C = \frac{V_{peak}X_C}{R} = \frac{V_{peak}}{2 \pi fCR}$$
Now we need to relate the impedance of the circuit to the resistance of the coil:
$$Z = \sqrt{R^2 + X^2} \Rightarrow R = \sqrt{Z^2 - X^2}$$
We substitute for R:
$$V_C = \frac{V_{peak}}{2 \pi fC \sqrt{Z^2 - X^2}}$$
Finally substitute numerical values and evaluate:
$$V_C = \frac{100 V}{2 \pi (60s^{-1})(332 \mu F)\sqrt{(10 \Omega)^2 - (8.0 \Omega)^2}} = \fbox{0.13 kV}$$

\paragraph{}
68 - $V_C = I_{rms}X_C$ and $I_{rms}$ from the potential difference across the inductor can serve to find C. If the resistance in the circuit vanishes, the measured rms voltage across both the capacitor and inductor is $V = \lvert V_L - V_C \rvert$.\\
(a) We find a relationship between the capacitance C and the potential difference across the capacitor:
$$V_C = I_{rms}X_C = \frac{I_{rms}}{2 \pi fC} \Rightarrow C = \frac{I_{rms}}{2 \pi fV_C}$$
Using the potential difference across the inductor to express the rms current in the circuit:
$$I_{rms} = \frac{V_L}{X_L} = \frac{V_L}{2 \pi fL}$$
Substituting for the $I_{rms}$ and using numerical values:
$$C = \frac{V_L}{(2 \pi f)^2LV_C} = \frac{50 V}{[2 \pi (60s^{-1})]^2(0.25 H)(75 V)} = \fbox{$19 \mu F$}$$
(b) We need to express the measured rms voltage V across both the capacitor and the inductor when $R = 0$:
$$V = \lvert V_L - V_C \rvert$$
Finally we substitute numerical values and evaluate:
$$V = \lvert 50 V - 75 V \rvert = \fbox{25 V}$$

\paragraph{}
69 - The potential differences across each of the circuit elements can be found by the rms current in the circuit. Phasor diagrams and our knowledge of the phase shifts between the voltages across the three circuit elements to find the voltage diferrences across their combinations.\\
(a) We do express the potential diffenrence between points A and B in terms of $I_{rms}$ and $X_L$:
\begin{equation}\label{voltab}
  V_{AB} = I_{rms}X_L
\end{equation}
Expressing $I_{rms}$ in terms of $\mathcal{E}$ and Z:
$$I_{rms} = \frac{\mathcal{E}}{Z} = \frac{\mathcal{E}}{\sqrt{R^2 + (X_L - X_C)^2}}$$
We evaluate $X_L$ and $X_C$ to obtain:
$$X_L = 2 \pi fL = 2 \pi (60s^{-1})(137 mH) = 51.648 \Omega$$
and
$$X_C = \frac{1}{2 \pi fC} = \frac{1}{2 \pi (60s^{-1})(25 \mu F)} = 106.10 \Omega$$
We evaluate for $I_{rms}$:
$$I_{rms} = \frac{115 V}{\sqrt{(50 \Omega)^2 + (51.648 \Omega - 106.10 \Omega)^2}} = 1.5556 A$$
Substitute numerical values in equation \ref{voltab} and evaluate:
$$V_{AB} = (1.5556 A)(51.648 \Omega) = 80.344 V = \fbox{80V}$$
(b) We express the potential difference between points B and C in terms of $I_{rms}$ and R:
$$V_{BC} = I_{rms}R = (1.5556 A)(50 \Omega) = 77.780 V = \fbox{78 V}$$
(c) Expressing the potential difference between points C and D in terms of $I_{rms}$ and $X_C$:
$$V_{CD} = I_{rms}X_C = (1.5556 A)(106.10 \Omega) = 165.05 V = \fbox{0.17 kV}$$
(d) The voltage across the inductor leads the voltage across the resistor as shown in the phasor diagram. We can use the Pythagorean theorem to find $V_{AC}$
$$V_{AC} = \sqrt{V_{AB}^2 + V_{BC}^2} = \sqrt{(80.0 V)^2 + (77.780 V)^2} = 111.58 V = \fbox{0.11 kV}$$
(e) The voltage across the capacitor lags the voltage across the resistor as shown in the phasor diagram and use the Pythagorean to find $V_{BD}$:
$$V_{BD} = \sqrt{V_{CD}^2 + V_{BC}^2}$$
$$= \sqrt{(165.05 V)^2 + (77.780 V)^2} = \fbox{182.46 V}$$

\paragraph{}
70 - We can find the power supplied to the circuit by $P_{av} = \mathcal{E}_{rms}I_{rms}\cos \delta$ and the resistance can be found by $P_{av} = I_{rms}^2R$. In (c) the impedance, inductive reactance, and resistance can be related to the capacitive reactance and solve for the capacitance C. We can use the condition on $X_L$ and $X_C$ at resonance to find the capacitance or inductance you would need to add to the circuit to make the power factor equal to 1.\\
(a) We begin by expressing the power supplied to the circuit in terms of $\mathcal{E}_{rms}, I_{rms}$ and the power factor $\cos \delta$ and substitute numerical values:
$$P_{av} = \mathcal{E}_{rms}I_{rms} \cos \delta = (120 V)(11 A) \cos 45^{\circ} = 933 W$$
(b) We relate the resistance to the power dissipated in the circuit:
$$P_{av} = I_{rms}^2 \Rightarrow R = \frac{P_{av}}{I_{rms}^2}$$
Substituting numerical values and evaluating:













\end{document}
